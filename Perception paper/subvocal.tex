\subsection{Sub-Vocalization}
When children learn to read, they typically do this by associating printed symbols with previously-acquired listening and speaking skills \cite{bruinsma_should_1980}. Even though this "silent reading" soon becomes inaudible, the movements of lips, tongue and larynx sometimes continue for many years. These small movements are known as \textit{sub-vocalization} \cite{bruinsma_should_1980}. It has been found that poor readers demonstrate a greater degree of sub-vocalization than proficient readers. It has been believed that lip movements and sub-vocalization should be suppressed, so their reading rates will increase \cite{bruinsma_should_1980}. However, according to \citeA{j_covert_1970}, oral behaviour during silent performance of language tasks does indeed serve a language function. \citeauthor{j_covert_1970} concludes that increased covert oral behaviour is beneficial during silent reading: \emph{"The implications for the teacher, thus, is that she should not tamper with the child's subvocalization? It is likely that the child NEEDS to subvocalize while reading, in any event, the subvocalization naturally becomes reduced in time."} \cite{bruinsma_should_1980}. Even if there are ways to reduce sub-vocalization, it might not be desirable for readers that have difficulty comprehending even low-level material. Instead, they should be encouraged to sub-vocalize to ensure maximum comprehension \cite{bruinsma_should_1980}.

That being said, sub-vocalization limits the speed of the reader, as \citeA{sub_speed} states: \emph{"When you voice your words as you read them as you do when you subvocalize, it means that you can only read as fast as you can talk out loud.  For most readers, that's only about 150 words per minute; a reading rate that puts you in the category of a slow reader."}

%http://www.jstor.org/stable/20195232
%[Conclusions The evidence cited here should caution teachers to be very careful in their efforts to reduce subvocalization during silent reading. This should be done directly only with students who are otherwise compe tent readers and who are reading relatively easy material ...]
