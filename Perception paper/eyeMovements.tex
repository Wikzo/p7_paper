\section{Background Knowledge}

\subsection{Eye movements}
To understand how speed reading is possible, it's important to know some of the basics behind how the eyes move.

When you read, visually analyse or look for something, your eyes are doing a series of movements called \textit{saccades}. In between these movements your eyes shortly fixate on elements, these stops called \textit{fixations}. Each fixations lasts only around 200-300 ms, so our eyes are quickly looking around a scene to find new details to fixate on. Luckily the movements of the eyes themselves are incredibly fast, reaching speeds of 500 degrees a second \cite{eyeMovement}.

But this all depends on what you're using your eyes for. There is generally three types of saccades: pursuit, vergence and vestibular. \textit{Pursuit} is when your eyes are trying to fixate on something moving. Generally, in these cases, the saccades are slower, as your eyes are following the target and not going back and forth between different targets \cite{eyeMovement}.
\textit{Vergence} is the inwards movement of your eyes to focus on something getting closer to you \cite{eyeMovement}.

\textit{Vestibular} movements happen when the eyes rotate to compensate for body or head movements. This is caused by visual stimulation, but mostly by the vestibular organ in the ears. This is also also why your sight gets blurry when you're dizzy.

In the case of reading, it's also important to talk about the different kinds of small saccades the eyes are capable of. \textit{Nystagmus} is very tiny and quick movements in the eyes, which causes a kind of tremor in your vision \cite{eyeMovement}. You will notice it when staring intently at a fixed point. It's believed that this is a precaution in the eyes to make the nerve cells keep firing by continuously stimulating them. Furthermore, the eyes also experience small \textit{drifts}. These are believed to be the results of a less-than-perfect control of the oculomotor resulting in your eyes slowly drifting to one side. To accommodate this the eyes make tiny saccades, called \textit{microsaccades}, to realign themselves \cite{eyeMovement}.

Further, it's important to the understanding of reading is \textit{saccade latency}. Every time a saccade is made, some calculations are needed to approximate where to move the eyes to fixate on a desired target. Even if excluding the uncertainty of where to move the eyes, it would still take 150-175 ms for the initial "request" of moving the eyes to the actual start of the movement. Cognitive processes further increases this latency, but also increases accuracy, meaning your fixation falls closer to the point of interest.

%\textbf{(Gustav: maybe write something about cognitive processes from Perception book here)}

During the saccade, though, everything is a blur. Or it should be anyway. It seems certain parts associated with processing visual inputs are halted during saccades. This process is called \textit{saccadic suppression}. But this is independent of the lexical processing, which means a reader is able to process read text in parallel with saccades \cite{eyeMovement}.

A fixation is not needed to read a word though. Your attention can be shifted to objects or words in your peripheral vision. But it is not possible to make a saccade without shifting attention to the fixation.