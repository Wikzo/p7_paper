\documentclass[jou,apacite]{apa6}
%\usepackage[style=apa,backend=biber]{biblatex}

\title{Analysis of RSVP Speed Reading in Contemporary Literature}
%\shorttitle{Does Rapid Serial Visual Presentation (RSVP) make us faster readers?}

\fourauthors{Mathias K. Berthelsen}{Gustav Dahl}{Benjamin N. Overgaard}{Andreas M. Thomsen}
\fouraffiliations{Study no. 20115689}{Study no. 20113263}{Study no. 20113220}{Study no. 20113259}


%HAMINGWAY!!!!!!!!!!

\abstract{
%This paper will investigate the speed reading technique known as Rapid Serial Visual Presentation (RSVP), as well as how it relates to other topics, such as eye movement and modes for reading. The goal of this paper is to analyze contemporary theory and literature and discuss the different points of view in regards to speed reading.

Rapid Serial Visual Presentation is a speed reading technique that presents words quickly one at a time. This is done to eliminate the need of saccading. In order to increase the reading speed, it's important to examine the optimal recognition position, as well as the processing speed of different words. The goal of this paper is to analyse contemporary theories and literature to discuss the strengths and weaknesses of RSVP. Even though it might be possible to read very quickly, it raises the question of how much it impacts the reading comprehension.}

\rightheader{RAPID SERIAL VISUAL PRESENTATION}
\leftheader{MTA14737}

\begin{document}
\maketitle

% A category with the (minimum) three required fields
%\category{H.4}{Information Systems Applications}{Miscellaneous}
%A category including the fourth, optional field follows...
%\category{D.2.8}{Software Engineering}{Metrics}[complexity measures, performance measures]

%\terms{Perception, reading}

%\keywords{Speed reading, reading, RSVP, comprehension, eye saccades} % NOT required for Proceedings

\section{Introduction}
%BASED ON PRELIMIANARY RESEARCH, TOOLS SHOULD BE DEVELOPED WITH A MODULAR MINDSET AND ALLOW FOR TWEAKABLE PARAMTERS and stuff

%Describe basic concept of the product (context)
%We have a collaboration with TAW
%They make 3D point n click game
%Framing system
%Path system

There are many ways to design camera motions in games. Fundamentally, one can distinguish between \textit{cinematic sequences} and \textit{interactive gameplay}. These two are typically considered mutually exclusive, because cinematic sequences per definition is non-interactive \cite{haigh-hutchinson_real-time_2009}. However, it is possible to mix those two, so the camera can dynamically adapt to certain things happening in the game, such as game events and player input. This means that a sequence does not have to be viewed in exactly the same way every time.

This paper presents an approach to creating a tool for a specific group of artists whose main working domain is time-based animation. Creating a tool for these artists is challenging, since they have knowledge and experience working with movies, which is a time-based medium, whereas the game will be dynamic and interactive.  The tool is able to define camera movements in a game where the player character moves along pre-defined paths. Our camera tool, named the \textit{Framing-based Camera Tool} (FCT), was developed using the Unity game engine. It has been designed in cooperation with the artists using methods from participatory design. The main finding from this was that the artists are quite comfortable with using keyframing animations. FCT has been designed with this concept in mind.

%Games often require the ability to replay previous sequences of the gameplay. This can be used to replicate certain events to re-create the motion and visual state of objects in a scene \cite{haigh-hutchinson_real-time_2009}. This can be achieved by recording the rendering state of objects on a set amount of frames, and then use \textit{interpolation} to calculate the state of said objects. Interpolation is a method of inserting intermediate values into a set of data and makes it possible to take sampled data and generate new points in between \cite{haigh-hutchinson_real-time_2009}. Replaying of this data can be referred to as \textit{keyframing}. Keyframing of camera data requires position and orientation of the camera, together with a time interval between the samples \cite{haigh-hutchinson_real-time_2009}.

The outline of the paper is as follows: We will discuss the related work in the next section. In Section 3 we will describe how the artists contributed to the design of FCT. Section 4 goes through the design and implementation of FCT, while Section (5?) shows how we evaluated the tool. We end with a summary and future work in Section (6?).

%During the collaboration, it was decided that we should focus on developing a camera system for the game. This tool should empower the artists, so that the artists didn't have to worry about technical details. It should be simple to set up and function in a similar fashion as other 3D applications that the TAW students have been trained in during their three-year education. The camera tool was chosen, since it does not directly influence and interfere with the gameplay, making it easier for the other programmers to work directly on the game.

%Before we began designing and implementing the tool, we conducted several preliminary studies to get an understanding of how game development tools should be made. We visited two game companies (KnapNok Games and Unity Studios), as well as conducting an online survey to gather, information about game development tools. The key findings were that the tool should be developed in a modular fashion and allow the user to tweak as many parameters as deemed necessary. Additional notes from the studies can be found in \textbf{APPENDIX X}.



%Sometimes it might be necessary to put certain restrictions on where the player can move. An example of this could be a special "boss battle" where the player is confined in a restricted area. Typically, the camera would zoom out and focus on specific parts of this boss enemy (e.g. a weak point). The camera dynamically frames the scene in such a way so that the player and the enemy are visible at all times \cite{haigh-hutchinson_real-time_2009}.

%This project is based on a collaboration between Medialogy and a group of artists from The Animation Workshop (TAW) in Viborg. As their bachelor project, the students at TAW developed a 3D point 'n click game, \textit{FEELS}, for the iPad using the Unity game engine. The TAW project spanned two semesters (pre-production and production), whereas this Medialogy project lasted only the first semester. Two additional programmers have also been working full-time on the project.
\section{Background Knowledge}

\subsection{Eye movements}
To understand how speed reading is possible, it's important to know some of the basics behind how the eyes move.

When you read, visually analyse or look for something, your eyes are doing a series of movements called \textit{saccades}. In between these movements your eyes shortly fixate on elements, these stops called \textit{fixations}. Each fixations lasts only around 200-300 ms, so our eyes are quickly looking around a scene to find new details to fixate on. Luckily the movements of the eyes themselves are incredibly fast, reaching speeds of 500 degrees a second \cite{eyeMovement}.

But this all depends on what you're using your eyes for. There is generally three types of saccades: pursuit, vergence and vestibular. \textit{Pursuit} is when your eyes are trying to fixate on something moving. Generally, in these cases, the saccades are slower, as your eyes are following the target and not going back and forth between different targets \cite{eyeMovement}.
\textit{Vergence} is the inwards movement of your eyes to focus on something getting closer to you \cite{eyeMovement}.

\textit{Vestibular} movements happen when the eyes rotate to compensate for body or head movements. This is caused by visual stimulation, but mostly by the vestibular organ in the ears. This is also also why your sight gets blurry when you're dizzy.

In the case of reading, it's also important to talk about the different kinds of small saccades the eyes are capable of. \textit{Nystagmus} is very tiny and quick movements in the eyes, which causes a kind of tremor in your vision \cite{eyeMovement}. You will notice it when staring intently at a fixed point. It's believed that this is a precaution in the eyes to make the nerve cells keep firing by continuously stimulating them. Furthermore, the eyes also experience small \textit{drifts}. These are believed to be the results of a less-than-perfect control of the oculomotor resulting in your eyes slowly drifting to one side. To accommodate this the eyes make tiny saccades, called \textit{microsaccades}, to realign themselves \cite{eyeMovement}.

Further, it's important to the understanding of reading is \textit{saccade latency}. Every time a saccade is made, some calculations are needed to approximate where to move the eyes to fixate on a desired target. Even if excluding the uncertainty of where to move the eyes, it would still take 150-175 ms for the initial "request" of moving the eyes to the actual start of the movement. Cognitive processes further increases this latency, but also increases accuracy, meaning your fixation falls closer to the point of interest.

%\textbf{(Gustav: maybe write something about cognitive processes from Perception book here)}

During the saccade, though, everything is a blur. Or it should be anyway. It seems certain parts associated with processing visual inputs are halted during saccades. This process is called \textit{saccadic suppression}. But this is independent of the lexical processing, which means a reader is able to process read text in parallel with saccades \cite{eyeMovement}.

A fixation is not needed to read a word though. Your attention can be shifted to objects or words in your peripheral vision. But it is not possible to make a saccade without shifting attention to the fixation.
%\subsection{Attention}

A fixation on a given word is not needed in order to read it. Your attention can be shifted to targets in or outside your foveal vision \cite{eyeMovement}. It is even possible to focus your attention on multiple targets simultaneously \cite{simAttention}. But it is not possible to make a saccade without shifting attention to the fixation \cite{eyeMovement}.
\subsection{Optimal Recognition Position} \label{ORP}
The destination of each saccade when reading, or the fixation point, depends on the content of the sentence; sometimes, the eye also skip words by saccading past them. 80\% of the fixation points are on content words (nouns, verbs and adjectives) and the remaining 20\% are on articles, pronouns, and conjunctions \cite{eysenck_cognitive_2010}. The fixation point inside the words themselves, the \textit{optimal recognition position} (ORP), has an impact on how fast a reader can name the word they are looking at. Research has shown that the ORP is near the middle or slightly left of the middle \cite{oregan_optimal_1992, nazir_letter_1998, oregan_convenient_1984}. The added recognition time as the fixation point deviates from the ORP is a U-shaped curve (see Figure \ref{fig:ucurve}), with around 20 ms added for each letter of deviation. For example, when reading a five-letter word, the ORP is 3.

\begin{figure}[htbp]
\centering
\includegraphics[width=0.4\textwidth]{Pics/ucurve}
\caption{The time it took participants to name a word depending on their fixation point. This graph shows results for words of length 5, 7, 9, and 11 \protect\cite{oregan_convenient_1984}.}
\label{fig:ucurve}
\end{figure}

Removing the spacing between words decreases the reading speed with around 30\% \cite{eyeMovement}. Fixation does not fall on the ORP, but tends to the beginning of words. Adding spaces in long words, like long Danish or German compound words, increase the reading speed, even though it is grammatically incorrect. Likewise with putting in spaces in Thai, where there normally are no spaces \cite{eyeMovement}.
\subsection{Reading Processes}
\citeA{carver_reading_1992} presents five reading processes, also referred to as \textit{gears} (see Table \ref{fig:trace_cross}). Each gear is defined by its \textit{goal}, \textit{culminating component}, and its \textit{speed}, measured by \textit{words per minute} (WPM). WPM also varies based on reading level, so to block out this bias, \citeauthor{carver_reading_1992} bases his experiments exclusively on college students. The following section is based on his study.

\begin{figure*}[htbp]
\centering
\captionsetup{justification=centering}
\includegraphics[width=0.8\textwidth]{Pics/gears_list}
\caption{Each gear has its own goal, culminating component, and WPM. \protect\cite{carver_reading_1992}}
\label{fig:ucurve}
\end{figure*}

The goal relies on the reader's intent when reading the text. For instance, the reader could read the text just to find a single word (scanning). The culminating component of a gear, is then the cognitive process that it requires. \textit{Lexical access} is the component used for finding a single word in memory. If the reader's goal changed to finding a certain sentence in the text, such as in skimming, transposed words must be found and the activity would require an additional component; \textit{semantic encoding}. Other than just finding the words, the reader must now determine the meaning of the sentence. In order for the reader to understand the complete thought of a sentence, the \textit{sentential integration} component must be added. These three components together are the requirement for the most basic and most used reading process - \textit{rauding}, also known as \textit{typical reading}. The word comes from a combination of 'auding' and 'reading', as they both share the same underlying comprehension processes. If the goal of reading the text is to learn, e.g., in order to answer a test, the \textit{idea remembering} component is added. In this gear, some words require re-reading and longer time to process. The final gear is focused around remembering the text and uses the \textit{fact rehearsal} component. This gear requires rehearsing the material and memorizing it. 
\citeA{carver_reading_1992} goes on to mention that the best readers shift up and down in gear while reading, based on the difficulty of the material - a term called \textit{process flexibility}. 

By having college students read a text followed by answering two multiple choice tests, as well as judging their own performance, \citeauthor{carver_reading_1992} tested efficiency at different reading rates. It was found that the students were most efficient at rates around 300 WPM - the rauding reading rate. This rate has therefore been referred to as the most optimal reading rate.
%Efficiency was calculated from the product of accuracy and reading rate, and the accuracy was determined through multiple-choice tests as well as self-judgement.

\citeauthor{carver_reading_1992} also mentions the term \textit{cognitive speed}, which acts as a limit of the reading speed. If the reader passes this limit, he will not be able to operate the culminating components successfully, resulting in a poor comprehension. The main concern however for most students is to make the reading speed reach the limit of the cognitive speed.

\citeauthor{carver_reading_1992} describes how \textit{rauding theory} relates to the \textit{schema theory}. Schema theory uses gears 1 and 2, but mainly focuses on how readers learn or memorize text. \citeA{widmayer_schema_2005} presents schema as a set of rules that help processing new information by interpreting and predicting situations occurring in the environment. Specifically for reading, \citeauthor{widmayer_schema_2005} claims that using schema allows learners to process information more effectively while reading. A common reading strategy that incorporates this is making predictions of the text before reading it by for instance reading the titles and looking at the pictures first.

This might indicate that when using schema, readers use skimming or scanning before reading a text, but shift to the memorization and learning gears as soon as they start reading. It also indicated that a complete overview of the text can be relevant in some cases.

%(INSERT REF: Reading for One Second, One Minute, or One Year) compares four different theoretical perspectives in reading, each focusing on a certain style and speed of reading: Rauding, Verbal Efficiency, Schema, and Whole Language.
\subsection{Sub-Vocalization}
When children learn to read, they typically do this by associating printed symbols with previously-acquired listening and speaking skills \cite{bruinsma_should_1980}. Even though this "silent reading" soon becomes inaudible, the movements of lips, tongue and larynx sometimes continue for many years. These small movements are known as \textit{sub-vocalization} \cite{bruinsma_should_1980}. It has been found that poor readers demonstrate a greater degree of sub-vocalization than proficient readers. It has been believed that lip movements and sub-vocalization should be suppressed, so their reading rates will increase \cite{bruinsma_should_1980}. However, according to \citeA{j_covert_1970}, oral behaviour during silent performance of language tasks does indeed serve a language function. \citeauthor{j_covert_1970} concludes that increased covert oral behaviour is beneficial during silent reading: \emph{"The implications for the teacher, thus, is that she should not tamper with the child's subvocalization? It is likely that the child NEEDS to subvocalize while reading, in any event, the subvocalization naturally becomes reduced in time."} \cite{bruinsma_should_1980}. Even if there are ways to reduce sub-vocalization, it might not be desirable for readers that have difficulty comprehending even low-level material. Instead, they should be encouraged to sub-vocalize to ensure maximum comprehension \cite{bruinsma_should_1980}.

That being said, sub-vocalization limits the speed of the reader, as \citeA{sub_speed} states: \emph{"When you voice your words as you read them as you do when you subvocalize, it means that you can only read as fast as you can talk out loud.  For most readers, that's only about 150 words per minute; a reading rate that puts you in the category of a slow reader."}

%http://www.jstor.org/stable/20195232
%[Conclusions The evidence cited here should caution teachers to be very careful in their efforts to reduce subvocalization during silent reading. This should be done directly only with students who are otherwise compe tent readers and who are reading relatively easy material ...]

\section{Speed Reading}
GUSTAV

Speed reading is the act of trying to read faster than normal. There are various ways to read a given text, depending on the context and the purpose of the reading \cite{differentWaysOfReading}. \citeA{ziefle_effects_1998} has shown than when reading on paper, people read an average of 201 words per minute, and about 180 when reading on a monitor (depending on its resolution). Speed reading is all about raising one's words per minute. Numerous techniques have been utilized throughout the years, and  thanks to computers, reading software is becoming more available these days.


\subsection{Rapid Serial Visual Presentation (RSVP)}
GUSTAV

Rapid Serial Visual Presentation is a technique that has become popular in the few last years. It's the idea of presenting words in small flashes, one at a time. In traditional reading, jumping from one word to another is done by saccading, which has a time penalty, since the eyes physically have to move back and forth. In RSVP, the goal is to eliminate saccading, thereby increasing reading speed. One of the more popular RSVP solutions is Spritz. According to their website, about 80\% of the time spent reading is used on physically moving the eyes from word to word \cite{spritz}.	It is claimed that by utilizing RSVP, it is possible to reduce this time. Additionally, by aligning the words according to the optimal recognition position, results can get even faster. Figure \ref{fig:spritz_orp} illustrates this concept.

\begin{figure}[htbp]
\centering
\includegraphics[width=0.4\textwidth]{Pics/opr_spritz}
\caption{Spritz utilizes the optimal recognition position to align words.}
\label{fig:spritz_orp}
\end{figure}

\subsection{Critiques of RSVP}
GUSTAV

In order to test how efficient RSVP actually is, a study was conducted by \citeA{schotter_dont_2014}. One thing that is criticized with RSVP techniques is the ability to go back and re-read words for improving one's comprehension. This process is called \textit{regression}, and according to \citeauthor{schotter_dont_2014} about 10\% to 15\% of the time spent reading is by making regressions, i.e. moving eyes back in the text to read material that has previously been processed. The hypothesis is that regression supports reading comprehension, since it allows readers to access more information from the text. This is especially true with texts that are more difficult to read, i.e. that it requires the reader to go back and read words again to make sense of how the sentence is structured. Readers are more likely to make a regression when they sense that their comprehension of the sentence has faltered \cite{schotter_dont_2014}.

By using trailing-mask conditions in their experiment, meaning that whenever a person has read a word, it becomes unreadable by changing it to X's instead, \citeA{schotter_dont_2014} could measure the effects of RSVP (i.e., being shown one word at a time - see Figure \ref{fig:trace_cross}). They found evidence that support the importance of regression. They used a combination of simple sentences and ambiguous garden-path sentences (e.g., "While
the man drank the water that was clear and cold overflowed from the toilet") to investigate when participants tended to regress the most, as well as measure their reading comprehension by answering questions about the sentences. \citeauthor{schotter_dont_2014} found that restricting the opportunity to re-read words with the trailing-mask (replacing the letters with X's) decreased comprehension globally. Unsurprisingly, they also found that participants were significantly more likely to regress in ambiguous sentences, since they had to go back and re-read to understand the sentence properly. Participants made regressions to go back and search for information that would support their overall understanding of the given text. \citeauthor{schotter_dont_2014} also saw evidence that suggests that participants were able to suppress the tendency to re-read, when they knew that regression was not possible due to the trailing-mask system.

(INSERT REF) suggests some improvements that should be made to RSVP. One improvement is that it should only present samples of the text, e.g. by selecting only the most informative words of a sentence. This could be determined by only presenting the least frequent content words as well as critical function words. (INSERT REF) also suggests that the duration each word is presented, should be based on an estimate of the processing time for the word. Spritz calculates an estimate by using the shape of the word as well as the length of the word(INSERT SPRITZ REF), however there might be different theories on how to calcuate this \textbf{(Maybe find a better way to write this)}.

\begin{figure}[htbp]
\centering
\includegraphics[width=0.45\textwidth]{Pics/trace_crosses}
\caption{Schotter et al. used eye-tracking to mask the words a person had already read, making it impossible to regress. Asterisks represent eye fixations.}
\label{fig:trace_cross}
\end{figure}

\citeauthor{schotter_dont_2014} conclude that reading without the ability to re-read parts of the text, decreases the comprehension accuracy and leads to a poorer understanding of the text.


Loss of spatial awareness and formatting

\subsection{Critiques of RSVP}
In order to test how efficient RSVP actually is, a study was conducted by \citeA{schotter_dont_2014}. One thing that is criticized with RSVP techniques is the ability to go back and re-read words for improving one's comprehension. This process is called \textit{regression}, and according to \citeauthor{schotter_dont_2014} about 10\% to 15\% of the time spent reading is by making regressions, i.e. moving eyes back in the text to read material that has previously been processed. The hypothesis is that regression supports reading comprehension, since it allows readers to access more information from the text. This is especially true with texts that are more difficult to read, i.e. that it requires the reader to go back and read words again to make sense of how the sentence is structured. Readers are more likely to make a regression when they sense that their comprehension of the sentence has faltered \cite{schotter_dont_2014}.

By using trailing-mask conditions in their experiment, meaning that whenever a person has read a word, it becomes unreadable by changing it to X's instead, \citeA{schotter_dont_2014} could measure the effects of RSVP (i.e., being shown one word at a time - see Figure \ref{fig:trace_cross}). They found evidence that support the importance of regression. They used a combination of simple sentences and ambiguous garden-path sentences (e.g., "While
the man drank the water that was clear and cold overflowed from the toilet") to investigate when participants tended to regress the most, as well as measure their reading comprehension by answering questions about the sentences. \citeauthor{schotter_dont_2014} found that restricting the opportunity to re-read words with the trailing-mask (replacing the letters with X's) decreased comprehension globally. Unsurprisingly, they also found that participants were significantly more likely to regress in ambiguous sentences, since they had to go back and re-read to understand the sentence properly. Participants made regressions to go back and search for information that would support their overall understanding of the given text. \citeauthor{schotter_dont_2014} also saw evidence that suggests that participants were able to suppress the tendency to re-read, when they knew that regression was not possible due to the trailing-mask system.

\begin{figure}[htbp]
\centering
\includegraphics[width=0.45\textwidth]{Pics/trace_crosses}
\caption{\protect\citeauthor{schotter_dont_2014} used eye-tracking to mask the words a person had already read, making it impossible to regress. Asterisks represent eye fixations. \protect\cite{schotter_dont_2014}}
\label{fig:trace_cross}
\end{figure}

\citeauthor{schotter_dont_2014} conclude that reading without the ability to re-read parts of the text, decreases the comprehension accuracy and leads to a poorer understanding of the text.

%\textbf{Loss of spatial awareness and formatting}

\citeA{rsvp_critique} suggests some improvements that should be made to RSVP. One improvement is that it should only present samples of the text, e.g. by selecting only the most informative words of a sentence. This could be determined by only presenting the least frequent content words as well as critical function words. \citeA{rsvp_critique} also suggests that the duration each word is presented, should be based on an estimate of the processing time for the word. \citeA{spritz_rhythm} claim they they calculate an estimate by using the shape of the word as well as the length of the word, however there might be different theories on how to calculate this.

%% by Gustav:
Another element that RSVP lacks is the sense of the layout and formatting of the text. Since you are only presented one word at a time, there is no spatial relationship between the words, as is the case when reading a book or a news paper.

\citeA{baker_is_2005} conducted a study to investigate different layouts when reading online. He found that depending on the reader and the context, different ways of formatting the text were preferred. Longer lines generally facilitate faster reading speeds, while shorter lines can result in increased reading comprehension. Additionally, reading speeds are faster for both single and multiple columns, but there is a preference for multiple short columns with 45-65 number of characters per line being the optimal. The results suggest that there is not a one best way to present text online, but that faster readers performed best when reading two-column texts. Slower readers benefited more from single-column layouts \cite{baker_is_2005}.
\section{Discussion and Conclusion}
When comparing the different reading processes to RSVP, an assumption can be made that RSVP only incorporates the rauding process. Neither skimming or scanning are supported, as they require peripheral vision. This also excludes the possibility of using schema, since readers do not have possibility of just reading titles or viewing pictures. Furthermore, idea remembering and fact rehearsal cannot be accomplished with RSVP, since they require regression. 

Since RSVP seems limited to only one reading process, process flexibility is mostly likely not supported either. Spritz claim however that their RSVP algorithm takes into account the processing time for each word when calculating the duration for each word. This may be a compensation for process flexibility. Whether or not the reading efficiency with this RSVP algorithm is the same as when using process flexibility remains to be tested.

<<<<<<< HEAD
It seems that the idea of RSVP is less well-suited for more complex texts. As stated earlier, sub-vocalization is an integral part of increasing reading comprehension. Since sub-vocalization follows the speed of how fast the reader is able to speak (about 150 words), it becomes difficult to sub-vocalize when being presented words at speeds of WPM rates of 500 or even 1000. Spritz and similar speed-reading apps might prove useful for less-complex material, but as stated by \citeA{time_spritz}, if we want to get a deeper understanding of a given text, \emph{"[R]eading with an app like Spritz allows us only to read simply, foolishly fast."}
=======
RSVP also claims to increase reading speeds since they are removing the need to saccade, since processing of visual inputs are halted, as described in \ref{eye}. However, since lexical processing runs in parallel 
to saccadic suppression, a reader can still process text during a saccade. Therefore, the benefits of RSVP might not be to the same degree as previously claimed.

Loss of spatial awareness and formatting
>>>>>>> 25caf95164f746028f85a5d31e901a5337b710fd


%%%% ---------- How to make headlines and sections ---------- %%%
\section{This is a section} \label{sec:thisSection}
\subsection{This is a subsection}
Let us refer to section \ref{sec:thisSection}.

%%% How to write bold, italics %%%
This text is \textbf{bold}.
This text is \textit{italics}.

%%% ---------- Insert page break ---------- %%%
%%\newpage
%%Here is some text on the next page

%%% ---------- This is how you refer to a figure in the text ---------- %%%
Here is something that I illustrate in figure \ref{fig:wavelength}.

%%% ---------- This is how you insert a single picture ---------- %%%
\begin{figure}[htbp]
\centering
\includegraphics[width=0.50\textwidth]{Pics/Dummy}
\caption{Image caption text goes here bla bla bla bla}
\label{fig:wavelength}
\end{figure}

%%% ---------- This is how you insert multiple pictures ---------- %%%
\begin{figure}[htbp] \centering
\begin{minipage}[b]{0.45\textwidth} \centering
\includegraphics[width=0.60\textwidth]{Pics/Dummy} % Venstre billede
\end{minipage} \hfill
\begin{minipage}[b]{0.45\textwidth} \centering
\includegraphics[width=0.60\textwidth]{Pics/Dummy} % Højre billede
\end{minipage} \\ % Captions og labels
\begin{minipage}[t]{0.45\textwidth}
\caption{Caption text for left picture.} % Venstre caption og label
\label{fig:cap1}
\end{minipage} \hfill
\begin{minipage}[t]{0.45\textwidth}
\caption{Caption text for right picture} % Højre caption og label
\label{fig:cap2}
\end{minipage}
\end{figure}

%%% ---------- This is how to make a source reference ---------- %%%
According to bla bla \cite{haigh-hutchinson_real-time_2009} %% passive source
at cite flere: \cite{haigh-hutchinson_real-time_2009, haigh-hutchinson_real-time_2009, haigh-hutchinson_real-time_2009}


%%% ---------- This is how to make bullet points ---------- %%%
\begin{itemize}
\item \textbf{Wavelenght} - Measured in meters from wave top to wave top and denoted as $\lambda$.
\item \textbf{Frequency} - Measured in oscillations per second, Hz, denoted $f$.
\item \textbf{Energy} - Measured in electronvolts, eV, denoted $E$.
\end{itemize}

%%% ---------- This is how to do math stuff ---------- %%%
To derive the wavelength or the frequency, formula \ref{eq:wavelenght} is applied:
\begin{align}
\centering 
\lambda = \frac{C}{f}
\label{eq:wavelenght} 
\end{align}
where {$C$} is the speed of light.



%%% This is how to make footnotes %%%
Hello, I need a footnote \footnote[0]{You can read me, no?}.

%%% This is how to insert a table %%%
\begin{table}[htbp]
\centering
\begin{tabular}{|l|c|c|}
\hline
& Personer
& Totalpris \\\hline
Lasagne
& 4
& 160
\\\hline
Flødekartofler
& 6
& 210
\\\hline
\end{tabular}
\caption{Valg af mad.}
\label{tab:mums}
\end{table} %% GUSTAV'S GUIDE: look at this for how to insert figures, quotes, etc.
\bibliography{references}
%\bibliography{sample}
\end{document}
