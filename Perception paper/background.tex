\section{Background Knowledge}

\subsection{Eye movements}
MATHIAS

\subsubsection{Saccades}
MATHIAS

(velocity, fixations durations, latency)
Different types of saccades (pursuit - cricket example, vergence, vestibular, etc.)
Habitual eye movements (apping)
\subsection{Optimal Recognition Position} \label{ORP}
ANDREAS

As mentioned previously, the eye does not move smoothly across a sentence when reading, instead they saccades from word to word. The destionation of each saccade, or the fixation point, depends on the content of the sentence, sometime the eye also skip words by saccading past them. 80\% of the fixation points are on content words (nouns, verbs and adjvectives) and the remaining 20\% are on articles, pronouns, and conjunctions \cite{eysenck_cognitive_2010}. The fixation point inside the words themselves, also called Optimal Recognition Position (ORP), has an impact on how fast a reader can name the word they are looking at. Research has shown that the ORP is near the middle or slightly left of the middle \cite{oregan_optimal_1992, nazir_letter_1998, oregan_convenient_1984}. The added recognition time as the fixation point deviates from the ORP is a U-shaped curve, with around 20 ms added for each letter of deviation.


\subsection{Meta-guiding reading (using a pen to keep focus)}
GUSTAV

\subsection{Attention}
MATHIAS

\subsection{Modes for reading}
BENJAMIN

("gears" - depending on context, you switch "gear"):
Read for memorize
Read for learning
Rauding (sentential integration, lexical + semantic) - most optimal
Skimming (semantic encoding)
Scanning (lexical access? using memory)
Reading rate (WPM) - rauding is the best?
Cognitive speed vs. reading speed
E = AR (E: Efficiency, A: Accuracy, R: Rate)

\subsection{Sub-vocalization}
GUSTAV