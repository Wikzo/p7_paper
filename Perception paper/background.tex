\section{Background Knowledge}

\subsection{Eye movements}
To understand how speed reading is possible, it's important to understand some basics on how the eye moves and

When you read, visually analyse or look for something, your eyes are doing a series of movements called \textit{saccades}. In between these movements your eyes shortly fixate on elements, these stops called \textit{fixations}. Each fixations lasts only around 200-300ms, so our eyes are quickly looking around a scene to find new details to fixate on. Luckily the movements of the eyes themselves are incredibly fast, reaching speeds of 500 degrees a second.

But this all depends on what you're using your eyes for. There is generally three types of saccades:
Pursuit, vergence and vestibular.
\textit{Pursuit} is when your eyes are trying to fixate on something moving. Generally in these cases your saccades are slower as your eyes are following the target and not going back and forth between different targets.
\textit{Vergence} is the inwards movement of your eyes to focus on something getting closer to you.
\textit{Vestibular} movement is the eyes rotating to compensate for body and/or head movement. This is both caused by visual stimulation, but mostly by the vestibular organ in your ears (which is also why your sight gets blurry when you're dizzy.)

In the case of reading, it's also important to talk about the different kinds of small saccades the eyes are capable of. \textit{Nystagmus} is very tiny and quick movements in the eyes, which causes are kind of tremor in your vision. You will notice it when staring intently at a fixed point. It's believed that this is a precaution in the eyes to make the nerve cells keep firing by continuously stimulating them. Furthermore, the eyes also experience small \textit{drifts}. These are believed to be the results of a less-than-perfect control of the oculomotor resulting in your eyes slowly drifting to one side. To accommodate this the eyes make tiny saccades, called \textit{microsaccades}, to realign themselves.

Further important to the understanding of reading is \textit{saccade latency}. Every time a saccade is made, some calculations are needed to approximate where to move the eyes to fixate on a desired target. Even if excluding the uncertainty of where to move the eyes, it would still take 150-175ms for the initial "request" of moving the eyes to the actual start of the movement. Cognitive processes further increases this latency, but also increases accuracy, meaning your fixation falls closer to the point of interest.

During the saccade, though, everything is a blur. Or it should be anyway. It seems certain parts associated with processing visual inputs are halted during saccades. This process is called \textit{saccadic suppression}. But this is independent of the lexical processing, which means a reader is able to process the words even during saccades.

\subsection{Optimal Recognition Position} \label{ORP}
ANDREAS

As mentioned previously, the eye does not move smoothly across a sentence when reading, instead they saccades from word to word. The destionation of each saccade, or the fixation point, depends on the content of the sentence, sometime the eye also skip words by saccading past them. 80\% of the fixation points are on content words (nouns, verbs and adjvectives) and the remaining 20\% are on articles, pronouns, and conjunctions \cite{eysenck_cognitive_2010}. The fixation point inside the words themselves, also called Optimal Recognition Position (ORP), has an impact on how fast a reader can name the word they are looking at. Research has shown that the ORP is near the middle or slightly left of the middle \cite{oregan_optimal_1992, nazir_letter_1998, oregan_convenient_1984}. The added recognition time as the fixation point deviates from the ORP is a U-shaped curve, with around 20 ms added for each letter of deviation.


\subsection{Meta-guiding reading (using a pen to keep focus)}
GUSTAV

\subsection{Attention}
MATHIAS

\subsection{Modes for reading}
BENJAMIN

("gears" - depending on context, you switch "gear"):
Read for memorize
Read for learning
Rauding (sentential integration, lexical + semantic) - most optimal
Skimming (semantic encoding)
Scanning (lexical access? using memory)
Reading rate (WPM) - rauding is the best?
Cognitive speed vs. reading speed
E = AR (E: Efficiency, A: Accuracy, R: Rate)

\subsection{Sub-vocalization}
GUSTAV