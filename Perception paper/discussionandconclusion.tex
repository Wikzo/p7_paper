\section{Discussion and Conclusion}
When comparing the different reading processes to RSVP, an assumption can be made that RSVP does not allow all reading processes to be used. Skimming and scanning can potentially be used by RSVP, as they only require the ability to recognize single words and transposed words. Rauding requires that the thought of a sentence can be understood, which requires that the reader has time to process the sentence before moving on. If the text is read at rate around the rauding rate - 300 WPM, it should be possible to do with RSVP.   Idea remembering and fact rehearsal however, cannot be accomplished with RSVP, since they require regression. The possibility of using schema is also excluded, since readers do not have possibility of just reading titles or viewing pictures.

Since RSVP seems to incorporate three different reading processes, process flexibility may be a possibility. According to \citeA{spritz}, their RSVP algorithm takes into account the processing time for each word, indicating further that it uses process flexibility. Whether or not the reading efficiency with this RSVP algorithm is the same as when using process flexibility without RSVP remains to be examined.

\citeA{spritz} also claims to increase reading speeds since they are removing the need to saccade, since processing of visual inputs are halted. However, since lexical processing runs in parallel 
to saccadic suppression, a reader can still process text during a saccade. Therefore, the benefits of RSVP might not be to the same degree as previously claimed.

It seems that the idea of RSVP is less well-suited for more complex texts. Sub-vocalization is an integral part of increasing the reading comprehension. Since sub-vocalization follows the speed of how fast the reader is able to speak (about 150 words), it becomes difficult to sub-vocalize when being presented words at a rate of 500 or even 1000 words per minute. 

Spritz and similar speed-reading apps might prove useful for less-complex material, but as stated by \citeA{time_spritz}, if a deeper understanding of a given text is needed, \emph{"reading with an app like Spritz allows us only to read simply, foolishly fast."}

