\section{Discussion and Conclusion}
When comparing the different reading processes to RSVP, an assumption can be made that RSVP only incorporates the rauding process. Neither skimming or scanning are supported, as they require peripheral vision. This also excludes the possibility of using schema, since readers do not have possibility of just reading titles or viewing pictures. Furthermore, idea remembering and fact rehearsal cannot be accomplished with RSVP, since they require regression. 

Since RSVP seems limited to only one reading process, process flexibility is mostly likely not supported either. Spritz claim however that their RSVP algorithm takes into account the processing time for each word when calculating the duration for each word. This may be a compensation for process flexibility. Whether or not the reading efficiency with this RSVP algorithm is the same as when using process flexibility remains to be tested.

RSVP also claims to increase reading speeds since they are removing the need to saccade, since processing of visual inputs are halted, as described in \ref{eye}. However, since lexical processing runs in parallel 
to saccadic suppression, a reader can still process text during a saccade. Therefore, the benefits of RSVP might not be to the same degree as previously claimed.

Loss of spatial awareness and formatting