\section{Introduction}
%BASED ON PRELIMIANARY RESEARCH, TOOLS SHOULD BE DEVELOPED WITH A MODULAR MINDSET AND ALLOW FOR TWEAKABLE PARAMTERS and stuff

%Describe basic concept of the product (context)
%We have a collaboration with TAW
%They make 3D point n click game
%Framing system
%Path system

This project has been based on a collaboration between four master students from Medialogy at Aalborg University and six artists from The Animation Workshop (TAW) in Viborg. As their bachelor project, the students at TAW developed a 3D point 'n click game for iPad, developed using Unity, called \textit{FEELS}. The TAW project lasted for two semesters (pre-production and production), whereas this project only lasted for a single semester. Three additional programmers have also been working full-time on the project.

During the collaboration, it was decided that we should focus on developing a camera system for the game. This tool should empower the artists to make a game with the required aesthetics, without having to worry about technical details. It should be simple to set up and function in a similar fashion as other 3D applications that the TAW students have been trained in during their three-year education. The camera tool was chosen, since it does not directly influence and interfere with the gameplay, making it easier for the other programmers to work directly on the game.

Before starting actual implementation, several preliminary studies were conducted to get an understanding of how game development tools should be made. We visited two game companies (KnapNok Games and Unity Studios), as well as conducting an online survey to gather information about game development tools. The key findings were that the tool should be developed in a modular fashion and allow the user to tweak as many parameters as deemed necessary. Additional notes from the studies can be found in \textbf{APPENDIX X}.