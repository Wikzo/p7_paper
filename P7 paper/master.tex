\documentclass{acm_proc_article-sp}

\begin{document}

\title{Camera Tool in Unity for CG Artists and Animators}
\subtitle{A Collaboration with The Animation Workshop}

\numberofauthors{4}
\author{
% 1st. author
\alignauthor
Mathias K. Berthelsen\\
       \affaddr{Aalborg University, Denmark}\\
       \email{mkbe11@student.aau.dk}
% 2nd. author
\alignauthor
Gustav Dahl\\
       \affaddr{Aalborg University, Denmark}\\
       \email{gdahl11@student.aau.dk}
% 3rd. author
\and
\alignauthor
Benjamin N. Overgaard\\
       \affaddr{Aalborg University, Denmark}\\
       \email{boverg11@student.aau.dk}
  % use '\and' if you need 'another row' of author names
% 4th. author
\alignauthor
Andreas M. Thomsen\\
       \affaddr{Aalborg University, Denmark}\\
       \email{amth11@student.aau.dk}
}

\maketitle
\begin{abstract}
Camera control in games is important, for example to create cinematic graphic effects or to guide the attention of the player. Unlike the camera in pre-rendered movies, a game camera should adapt to the player's movements. This paper describes the design of a tool that allows artists to frame the camera in games where the player moves along a pre-defined path. To this end, we used participatory design methods to understand how artists typically work with computer animations and found that they prefer to work with keyframing animation. This concept has been incorporated into the proposed Framing-based Camera Tool (FCT) in the form of framings. A framing consists of an influence point and a group of camera settings. Artists are able to define a group of framings by adjusting the position, orientation and field of view of the camera. Then, the game's camera interpolates between the framings automatically based on the player's position. Through an evaluation of our camera tool, we show that it allows artists to create and design dynamic camera animations.
\end{abstract}

% A category with the (minimum) three required fields
\category{H.4}{Information Systems Applications}{Miscellaneous}
%A category including the fourth, optional field follows...
\category{D.2.8}{Software Engineering}{Metrics}[complexity measures, performance measures]

\terms{Game development}

\keywords{Unity, game development, camera system, AI, interpolation, workflow, participatry design, collaboration} % NOT required for Proceedings

\section{Introduction}
%BASED ON PRELIMIANARY RESEARCH, TOOLS SHOULD BE DEVELOPED WITH A MODULAR MINDSET AND ALLOW FOR TWEAKABLE PARAMTERS and stuff

%Describe basic concept of the product (context)
%We have a collaboration with TAW
%They make 3D point n click game
%Framing system
%Path system

There are many ways to design camera motions in games. Fundamentally, one can distinguish between \textit{cinematic sequences} and \textit{interactive gameplay}. These two are typically considered mutually exclusive, because cinematic sequences per definition is non-interactive \cite{haigh-hutchinson_real-time_2009}. However, it is possible to mix those two, so the camera can dynamically adapt to certain things happening in the game, such as game events and player input. This means that a sequence does not have to be viewed in exactly the same way every time.

This paper presents an approach to creating a tool for a specific group of artists whose main working domain is time-based animation. Creating a tool for these artists is challenging, since they have knowledge and experience working with movies, which is a time-based medium, whereas the game will be dynamic and interactive.  The tool is able to define camera movements in a game where the player character moves along pre-defined paths. Our camera tool, named the \textit{Framing-based Camera Tool} (FCT), was developed using the Unity game engine. It has been designed in cooperation with the artists using methods from participatory design. The main finding from this was that the artists are quite comfortable with using keyframing animations. FCT has been designed with this concept in mind.

%Games often require the ability to replay previous sequences of the gameplay. This can be used to replicate certain events to re-create the motion and visual state of objects in a scene \cite{haigh-hutchinson_real-time_2009}. This can be achieved by recording the rendering state of objects on a set amount of frames, and then use \textit{interpolation} to calculate the state of said objects. Interpolation is a method of inserting intermediate values into a set of data and makes it possible to take sampled data and generate new points in between \cite{haigh-hutchinson_real-time_2009}. Replaying of this data can be referred to as \textit{keyframing}. Keyframing of camera data requires position and orientation of the camera, together with a time interval between the samples \cite{haigh-hutchinson_real-time_2009}.

The outline of the paper is as follows: We will discuss the related work in the next section. In Section 3 we will describe how the artists contributed to the design of FCT. Section 4 goes through the design and implementation of FCT, while Section (5?) shows how we evaluated the tool. We end with a summary and future work in Section (6?).

%During the collaboration, it was decided that we should focus on developing a camera system for the game. This tool should empower the artists, so that the artists didn't have to worry about technical details. It should be simple to set up and function in a similar fashion as other 3D applications that the TAW students have been trained in during their three-year education. The camera tool was chosen, since it does not directly influence and interfere with the gameplay, making it easier for the other programmers to work directly on the game.

%Before we began designing and implementing the tool, we conducted several preliminary studies to get an understanding of how game development tools should be made. We visited two game companies (KnapNok Games and Unity Studios), as well as conducting an online survey to gather, information about game development tools. The key findings were that the tool should be developed in a modular fashion and allow the user to tweak as many parameters as deemed necessary. Additional notes from the studies can be found in \textbf{APPENDIX X}.



%Sometimes it might be necessary to put certain restrictions on where the player can move. An example of this could be a special "boss battle" where the player is confined in a restricted area. Typically, the camera would zoom out and focus on specific parts of this boss enemy (e.g. a weak point). The camera dynamically frames the scene in such a way so that the player and the enemy are visible at all times \cite{haigh-hutchinson_real-time_2009}.

%This project is based on a collaboration between Medialogy and a group of artists from The Animation Workshop (TAW) in Viborg. As their bachelor project, the students at TAW developed a 3D point 'n click game, \textit{FEELS}, for the iPad using the Unity game engine. The TAW project spanned two semesters (pre-production and production), whereas this Medialogy project lasted only the first semester. Two additional programmers have also been working full-time on the project.
\section{Background Knowledge}

\subsection{Eye movements}
To understand how speed reading is possible, it's important to understand some basics on how the eye moves and

When you read, visually analyse or look for something, your eyes are doing a series of movements called \textit{saccades}. In between these movements your eyes shortly fixate on elements, these stops called \textit{fixations}. Each fixations lasts only around 200-300ms, so our eyes are quickly looking around a scene to find new details to fixate on. Luckily the movements of the eyes themselves are incredibly fast, reaching speeds of 500 degrees a second.

But this all depends on what you're using your eyes for. There is generally three types of saccades:
Pursuit, vergence and vestibular.
\textit{Pursuit} is when your eyes are trying to fixate on something moving. Generally in these cases your saccades are slower as your eyes are following the target and not going back and forth between different targets.
\textit{Vergence} is the inwards movement of your eyes to focus on something getting closer to you.
\textit{Vestibular} movement is the eyes rotating to compensate for body and/or head movement. This is both caused by visual stimulation, but mostly by the vestibular organ in your ears (which is also why your sight gets blurry when you're dizzy.)

In the case of reading, it's also important to talk about the different kinds of small saccades the eyes are capable of. \textit{Nystagmus} is very tiny and quick movements in the eyes, which causes are kind of tremor in your vision. You will notice it when staring intently at a fixed point. It's believed that this is a precaution in the eyes to make the nerve cells keep firing by continuously stimulating them. Furthermore, the eyes also experience small \textit{drifts}. These are believed to be the results of a less-than-perfect control of the oculomotor resulting in your eyes slowly drifting to one side. To accommodate this the eyes make tiny saccades, called \textit{microsaccades}, to realign themselves.

Further important to the understanding of reading is \textit{saccade latency}. Every time a saccade is made, some calculations are needed to approximate where to move the eyes to fixate on a desired target. Even if excluding the uncertainty of where to move the eyes, it would still take 150-175ms for the initial "request" of moving the eyes to the actual start of the movement. Cognitive processes further increases this latency, but also increases accuracy, meaning your fixation falls closer to the point of interest.

During the saccade, though, everything is a blur. Or it should be anyway. It seems certain parts associated with processing visual inputs are halted during saccades. This process is called \textit{saccadic suppression}. But this is independent of the lexical processing, which means a reader is able to process the words even during saccades.

\subsection{Optimal Recognition Position} \label{ORP}
ANDREAS

As mentioned previously, the eye does not move smoothly across a sentence when reading, instead they saccades from word to word. The destionation of each saccade, or the fixation point, depends on the content of the sentence, sometime the eye also skip words by saccading past them. 80\% of the fixation points are on content words (nouns, verbs and adjvectives) and the remaining 20\% are on articles, pronouns, and conjunctions \cite{eysenck_cognitive_2010}. The fixation point inside the words themselves, also called Optimal Recognition Position (ORP), has an impact on how fast a reader can name the word they are looking at. Research has shown that the ORP is near the middle or slightly left of the middle \cite{oregan_optimal_1992, nazir_letter_1998, oregan_convenient_1984}. The added recognition time as the fixation point deviates from the ORP is a U-shaped curve, with around 20 ms added for each letter of deviation.


\subsection{Meta-guiding reading (using a pen to keep focus)}
GUSTAV

\subsection{Attention}
MATHIAS

\subsection{Modes for reading}
BENJAMIN

("gears" - depending on context, you switch "gear"):
Read for memorize
Read for learning
Rauding (sentential integration, lexical + semantic) - most optimal
Skimming (semantic encoding)
Scanning (lexical access? using memory)
Reading rate (WPM) - rauding is the best?
Cognitive speed vs. reading speed
E = AR (E: Efficiency, A: Accuracy, R: Rate)

\subsection{Sub-vocalization}
GUSTAV
\section{Related Work}
%Camera Path Animator 3.0 (Unity asset store).
%Camera system that follows the character but also focuses on showing the environment - Used in the game God of War.

During researching, a tool called Camera Path Animator 3.0 by Jasper Stocker was found \cite{unity_camTool}. It's can be used for creating animated cameras within Unity (see Figure \ref{fig:unity_path_cam_tool}). As the name suggests, it works by animating the camera along a specified path, which can have various shapes (e.g., Bezier and Hermite curves). The tool is primarily targeted towards creating cameras that move linearly along a set path, i.e. for use in a non-interative cutscene. It provides various ways of inserting, moving and deleting points, as well as settings that can be changed, such as field of view, speed, interpolation type and easing. Additionally, it has a event system for triggering certain events at certain points in the path.


\begin{figure}[htbp]
\centering
\includegraphics[width=0.50\textwidth]{Pics/unity_path_cam_tool}
\caption{Overview of the objects related to the camera system.}
\label{fig:unity_path_cam_tool}
\end{figure}


Built specifically for animating cutscene cameras
Can be used to animate any gameobjects
Full source code included
Easy to use
Custom built into the Unity editor
Many animation modes; once, loop, reverse, reverse loop, ping pong
Many control modes; user rotation controlled, follow the path, follow a target, mouse look, reverse follow the path
Includes complete event system to communicate with your game
Mobile ready
PlayMaker compatible

%\section{Preliminary Study}
We did bla bla bla ...

\subsection{Participants}
	Interviews (Knapnok Games, Unity Studios)
	Online questionnaires (27 game developers)
	Observations (4-5 students at TAW)
	
	The interview sessions involved a total of three people from two game studios in Denmark; Project Lead at Knapnok Games in Copenhagen and Lead Designer and Lead Programmer at Unity Studios in Aarhus.
The online questionnaire received 27 submissions from people with a median professional experience of 3 years (SD 2.67).
Lastly, the participatory paper prototyping was done on 4 students at TAW. (from the team?!)

	
\subsection{Methods}
		Interview design (semi-structured)
		Questionnaire design
		Observations (from meetings, discussions)
		Paper prototyping
		Participatory design
		
		The interviews were conducted at the respective companie' offices using a semi-structured approach. The interviews was audio recorded and notes were taken during the interview. We were offered to see the tools they use in production in practice. The session at KnapNok Games took one hour and two hours at Unity Studios.
The questionnaires was published on relevant game development communities. (unity forum, tigsource, 3Dboss.com, game dev facebook groups and twitter). The questions in the questionnaire and the semi-structured interview was designed to be case-specific (hendrik bog) in hope of gathering more detailed information. Both the interviews and questionnaire was analysed with using grounded theory
The paper prototyping was a participatory design - dialogue-based prototyping session. The participants wrote down their requirements while thinking aloud. They then designed their "dream interface" using paper blocks. They expressed their decisions and ideas while doing so. A facilitator would ask why they needed certain elements and why it was chosen to structure it as they did.


\subsection{Findings}
	Frequently mentioned points
	
	The two game studios that were interviewed both use Unity in their development and mentioned that it is most efficient if everyone on the development team is able to use Unity. This allows team members non-programmers (e.g. artists and designers) to implement assets themselves as well as tweak parameters and build functionality using tools made by the programmers. This way, the amount of interactions between programmers and non-programmers are minimized, and non-programmers can take on more tasks.
Among the tools made for non-programmers was 
Points derived from the questionnaire mentions the more tweakable  your design is, the better, i.e. changing variables without going into the code in a text editor. However, this can also be of nuisance as the number of accessible variables can confuse them. A tool also have to be presented clearly. Finally it is important that the designers know the tools they are working with.
During the paper prototyping, the participants continuously mentioned Maya (the software package they're most accustomed to) when trying to explain how they had envisioned the camera tool working; they wanted something close to Maya.

%Participatory Design
%	What is PD?
%	What we did
%		Exercises
%	Findings
%		Keyframing concept
%			Interpolation design (position-based, not time-based)		
%		Workflow
%			Interface (base on their "comfort zone")
%			Be the cam - mental model
%			Aim point, etc.
%			Immediate feedback, etc.
%		
%
%	Mental model
%	Immediate feedback
%	Design iterations - listen to what they want, not what they say


%\section{Participatory Design}
%In the following, the users can also be described with the terms \textit{students} and \textit{artists}.

%NOTE: terms we use: scene camera vs framing camera
%Framing: influence point + camera + connection
%Player path

%An initial mockup of the tool's interface and functionality was created internally (see Figure \ref{fig:mockup}). The mockup had the interface all gathered in one window to keep the design minimalistic. However, in search of a better design, we included the artist in the process going forward.

%To ground our design work, we were interested in learning more about our users and how they envisioned the finished product. Therefore, three participatory exercises were conducted.\footnote{The exercises were video documented, except for one artist, and half of the footage from another artist in the first exercise.} The first was done in order to get a better understanding of the students' general workflow when working with virtual cameras in their tools they, namely Autodesk Maya and Unity. Secondly, we wanted to know which features they requested and how they envisioned the finished tool. Finally, they tried using a prototype of the camera tool based on their initial design and feature requests.

%\begin{figure}[htbp]
%\centering
%\includegraphics[width=0.30\textwidth]{Pics/InitialMockup}
%\caption{The initial concept was to have all the features inside one big window.}
%\label{fig:mockup}
%\end{figure}

%\subsection{Exercise 1: Knowledge of Tools} \label{exerciseOne}
%An artist and a facilitator sat down in front of the artist's work area. The facilitator then asked the artist to show him how he would animate a camera around an object in Maya (see Figure \ref{fig:mads_dual}). This included keyframing and how to change the camera settings for each keyframe. This was done to get an idea of how they usually work. If our camera tool should be successful at helping these artists, it should mimic some of the same behaviour that Maya has. It seemed ideal that the users should be able to utilize their previously-gained knowledge from Maya wherever applicable.

%\begin{figure}[htbp]
%\centering
%\includegraphics[width=0.3\textwidth]{Pics/Mads_dual}
%\caption{All the students at The Animation Workshop work with a dual-monitor setup.}
%\label{fig:mads_dual}
%\end{figure}

%After this, the facilitator and artist switched to Unity where he was tasked with moving around in the scene and then to create basic objects such as cubes and spheres, as well as positioning the camera. The purpose of this was to get an understanding of how skilled they are at using Unity - moving around the scene, rotating and translating objects etc. This was done with another two artists as well. The first section of the exercise, the Maya part, was the same for all three artists, but the questions and tasks changed in the last section as the facilitator gained new knowledge about their skill level.

%It was discovered that none of the artists knew how to use a standard Unity feature that lets the player fly around with the scene camera as if they were playing a first-person game ("Flythrough Mode", \cite{unity_flyMode}). The artist were excited about the discovery of this feature. One artist perceived the standard way of moving around in Unity as confusing, while another stated that the way of moving the camera is exactly like in Maya. After testing this ourselves, we concluded that the movement controls in Maya and Unity are indeed very similar (except for the "Flythrough Mode"), which means that the artists should ideally be comfortable with navigating in either of the applications.

%\subsection{Exercise 2: Paper Prototyping}
%The artists were given a blank sheet of paper and a pen and was tasked to draw and write about the camera tool, as they envisioned it in a free-form approach. They then explained what they did and they had a discussion about it. There were no strict requirements of how they approached the task, so some focused a lot on feature requires, while others were more interested in fundamental workflow and user interfaces. This was done with four artists in total.

%After this, the same four artists were placed at a desk, one by one, where a variety of labels were laid out in front of them (see Figure \ref{fig:labels}). These labels were marked as different sliders, buttons, windows, field parameters, camera settings, etc. The labels were primarily based on the basic building blocks of Unity's user interface, as well as some of the observations from the first exercise where the users talked about how they worked with Maya.

%\begin{figure}[htbp]
%\centering
%\includegraphics[width=0.3\textwidth]{Pics/labels}
%\caption{The users were asked to design their "dream program" by using paper labels as building blocks.}
%\label{fig:labels}
%\end{figure}

%All artists wanted basic features like translation, rotation, field of view of the camera, as well as a curve editor to change the interpolation between two cameras. After these features, all the artists diverted in their designs. All the designs were discussed internally and main points and ideas were extracted from them. These findings were used as the foundation of the tool's design. However, the foundation for the program was not strictly based on the artists ideas and requirements; instead, we tried to extract some more general requests of what they as artists actually need. It was important to keep in mind that working in an interactive medium such as game development was a new experience for the users.

%A slider that could be used to interpolate between two camera framings in a preview window is an example of an idea that was discarded by us. The artist wanted two preview windows of the camera framing, as well as a preview window of the interpolation, which the slider would manipulate. His idea was similar to a DJ turntable. After discussing it internally, we concluded that this idea would make the interface too elaborate, but the idea of being able to preview your changes quickly was kept.

% put into appendixs
%See Figures \ref{fig:morten_requirements} and \ref{fig:mads_turntable}.

% morten PUT INTO APPENDIX
%\begin{figure}[htbp]
%\centering
%\includegraphics[width=0.50\textwidth]{Pics/Morten01}
%\caption{Feature requirements.}
%\label{fig:morten_requirements}
%\end{figure}

% mads PUT INTO APPENDIX
%\begin{figure}[htbp]
%\centering
%\includegraphics[width=0.50\textwidth]{Pics/Mads01}
%\caption{Turntable concept.}
%\label{fig:mads_turntable}
%\end{figure}

%\subsection{Summary of Findings}
%It was found that the students are comfortable working in Maya and use it as their main reference point when working in 3D applications. The overall design of the camera tool should take this into considerations to make the transition from Maya to Unity as smooth as possible.

%Some of the more concrete findings include:

%\vspace{-5mm}
%\begin{itemize}
%\setlength\itemsep{0em}
%\item Basic camera manipulation: translation, rotation, field of view, lenses, easing, etc.
%\item Camera modes
%\begin{itemize}
%\item Look through the camera when positioning it
%\item Setup the position of the camera by making it aim at a 3D point in space
%\item Change \textit{Camera Settings} or behaviour when player enters a Domain areas (trigger zone)
%\item Toggle camera to either follow or be static
%\end{itemize}
%\item Curves and graph editors
%\item Hotkeys (e.g., set new camera with "S")
%\item Easy to get an overview and navigate between framings
%\item Tool should be optimal for dual-monitor setups and easy to customize window layout (draggability)
%\item Optimal workflow
%\begin{itemize}
%\item Real-time preview: Be able to move a puppet-like character around to get a feel of the interpolation (similar to the yellow Google "pegman" in Street View)
%\item Having an easy-to-access list of all framings
%\item 2D overview of the map with all camera keys/markers
%\item When looking through the camera, have transparent border (like in Maya)
%\item In camera preview, show what camera is currently 'dominant' (how much weight each camera is pulling)
%\end{itemize}
%\end{itemize}

%\subsection{Design and Implementation}
%After completing the two exercises, we started developing a rough prototype in Unity (see Figure \ref{fig:prototype}). This iteration included basic functionality to make it possible to test and get feedback as quickly as possible. When discussing the tool, it was found necessary to define some terms. 

%The camera system has to work with an existing path system that uses \textit{path nodes}. The camera system consists of two main objects, namely \textit{influence points} and \textit{camera settings} (or simply just \textit{cameras}). Together, these are called \textit{framings} and are what the artist is going to place. Figure \ref{fig:manual_overview} shows a simple illustration of this.

%The player character in \textit{FEELS} will move on a path-based system, therefore the camera tool is path-based as well. The system will allow the placement of \textit{influence points} along the player character's movement-path, each point is associated with a \textit{camera setting} (or simply just \textit{cameras}) with information about the camera's position, rotation, and field of view. Together, these are called \textit{framings} and are what the artist is going to place. Figure \ref{fig:manual_overview} shows a simple illustration of this. The system will then interpolate the main camera between two connected framings using the player's position on the path.

%\begin{figure}[htbp]
%\centering
%\includegraphics[width=0.40\textwidth]{Pics/manual_overview}
%\caption{Overview of the objects related to the camera system.}
%\label{fig:manual_overview}
%\end{figure}

%\subsubsection{General Restrictions}
%The director of the game project had some requirements for how the camera system should work. First of all, it was important that the player character is always visible on the screen. Additionally, the game will feature no traditional \textit{cuts}, so whenever the player is moving, the camera should follow without suddenly cutting to a new angle.

%\subsubsection{How the Interpolation Algorithm Works}
%Write a little more in-depth how influence points and interpolation actually works...
%Depending on the player's position relative to the influence points, the game's \textit{main camera} will inherit camera settings from the framing's cameras.



%\subsection{Exercise 3: First Iteration}
%During one of their daily SCRUM meetings, we presented the tool to the students, as well as providing them a short manual (SEE APPENDIX A). Afterwards, the prototype was shown to three of the students to provide feedback. Each test session lasted for about 10-40 minutes. The students got a chance to try out the tool, as well as expressing their overall opinion about how they preferred the workflow to be.

%The facilitator gave the artist small tasks like "Can you place framings along this paths?" and "Can you try and set the framing's camera settings as you'd like?". This exercise was semi-structured, and not all questions were predetermined; some was thought up as the exercise went on.

%\begin{figure}[htbp]
%\centering
%\includegraphics[width=0.40\textwidth]{Pics/MainSetup}
%\caption{Rough prototype with basic camera manipulation.}
%\label{fig:prototype}
%\end{figure}

%Over the course of the exercise, some key points was discovered through observation or discussion with the artist. The biggest piece of feedback revolved how the users are supposed to place the virtual camera in the scene. Initially, users were supposed to use Unity's "Flythrough Mode" (see Section \ref{exerciseOne}) and place the camera as they liked. Then, they should press the "Save camera position" button (see Figure \ref{fig:prototype}). However, some users were confused about this, since they thought that they now they continuously looked through the camera's point of view, i.e. if they moved around after they'd pressed the button, the framing camera should be changed as well. This was not how the system worked, since it simply just saved the scene camera's position to the framing camera whenever the button was pressed. In other words, the users mental models didn't fit with how the tool worked.

\section{Experiment}
%Give artists the task to make a video in a certain scene using some specific features, both in Unity and in Maya. We can then compare their performances, and if the two performances are significantly close, the tool was a success.

%Video and audio recordings. Time spent on tasks. Few questions about usability and experience.

%(Hypotheses - not sure yet?)

\subsection{Purpose of the test}
\begin{enumerate}
\item To find out if the FEELS team’s mental models of the camera system match how the system actually works.
\item Compare other artists from TAW 3rd year using the camera system with the FEELS team.
\end{enumerate}


Figure \ref{fig:test_overview} shows the criteria for the two testing groups \footnote{This will not be in the final paper; it's just for helping you to understand what we mean.}


\begin{figure}[htbp]
\centering
\includegraphics[width=0.5\textwidth]{Pics/test_table_temp}
\caption{Overview of the testing groups.}
\label{fig:test_overview}
\end{figure}

\subsection{The Three Phases of the Experiment}
\textbf{PHASE 1 - Training:}
\begin{enumerate}
\item Place framings along the movement path so that there's at least one framing in "move path section".
\item Tell the facilitator about the functionality of each button in the interface as you think it'll work.
\item Split a framing connection into two.
\item Delete a framing.
\end{enumerate}

Specific tasks:
\begin{enumerate}
\item Make the camera's field of view change when player gets close to it
\item Make the camera tilt upwards when player gets close to it
\item Make the camera pan to the left when player gets close to it
\item Make the camera look at object X when the player gets close to it
\item Make the camera dolly away from the character as the player walks along a path.
\item Make the camera look from the ground upwards
\item Change the interpolation of one of the previous assignments by changing the animation curve.
\end{enumerate}

\textbf{PHASE 2 - Re-create camera:}
\begin{itemize}
\item Show a video of the end result
\item Recreate this as closely as possible?
\end{itemize}

\textbf{PHASE 3 - Be creative:}
\begin{itemize}
\item Give them environment with path already there, do what you want. Test the limits of the system.
\end{itemize}

\section{Results}
What did we find in the 6. Experiment.
\section{Conclusion}
We found that ...
\section{Discussion}
\subsection{Principles and Guidelines}
General findings that can be learned from this study. Could be something like the "180 degree rule" of games. Examples: that the barycentric interpolation works better than triangulation; that you should at maximum have four cameras; that cutting between multiple cameras confuses the player if the cutting speed is less than X, etc. In short, something the reader can use after having read the paper.

%%%% ---------- How to make headlines and sections ---------- %%%
\section{This is a section} \label{sec:thisSection}
\subsection{This is a subsection}
Let us refer to section \ref{sec:thisSection}.

%%% How to write bold, italics %%%
This text is \textbf{bold}.
This text is \textit{italics}.

%%% ---------- Insert page break ---------- %%%
%%\newpage
%%Here is some text on the next page

%%% ---------- This is how you refer to a figure in the text ---------- %%%
Here is something that I illustrate in figure \ref{fig:wavelength}.

%%% ---------- This is how you insert a single picture ---------- %%%
\begin{figure}[htbp]
\centering
\includegraphics[width=0.50\textwidth]{Pics/Dummy}
\caption{Image caption text goes here bla bla bla bla}
\label{fig:wavelength}
\end{figure}

%%% ---------- This is how you insert multiple pictures ---------- %%%
\begin{figure}[htbp] \centering
\begin{minipage}[b]{0.45\textwidth} \centering
\includegraphics[width=0.60\textwidth]{Pics/Dummy} % Venstre billede
\end{minipage} \hfill
\begin{minipage}[b]{0.45\textwidth} \centering
\includegraphics[width=0.60\textwidth]{Pics/Dummy} % Højre billede
\end{minipage} \\ % Captions og labels
\begin{minipage}[t]{0.45\textwidth}
\caption{Caption text for left picture.} % Venstre caption og label
\label{fig:cap1}
\end{minipage} \hfill
\begin{minipage}[t]{0.45\textwidth}
\caption{Caption text for right picture} % Højre caption og label
\label{fig:cap2}
\end{minipage}
\end{figure}

%%% ---------- This is how to make a source reference ---------- %%%
According to bla bla \cite{haigh-hutchinson_real-time_2009} %% passive source
at cite flere: \cite{haigh-hutchinson_real-time_2009, haigh-hutchinson_real-time_2009, haigh-hutchinson_real-time_2009}


%%% ---------- This is how to make bullet points ---------- %%%
\begin{itemize}
\item \textbf{Wavelenght} - Measured in meters from wave top to wave top and denoted as $\lambda$.
\item \textbf{Frequency} - Measured in oscillations per second, Hz, denoted $f$.
\item \textbf{Energy} - Measured in electronvolts, eV, denoted $E$.
\end{itemize}

%%% ---------- This is how to do math stuff ---------- %%%
To derive the wavelength or the frequency, formula \ref{eq:wavelenght} is applied:
\begin{align}
\centering 
\lambda = \frac{C}{f}
\label{eq:wavelenght} 
\end{align}
where {$C$} is the speed of light.



%%% This is how to make footnotes %%%
Hello, I need a footnote \footnote[0]{You can read me, no?}.

%%% This is how to insert a table %%%
\begin{table}[htbp]
\centering
\begin{tabular}{|l|c|c|}
\hline
& Personer
& Totalpris \\\hline
Lasagne
& 4
& 160
\\\hline
Flødekartofler
& 6
& 210
\\\hline
\end{tabular}
\caption{Valg af mad.}
\label{tab:mums}
\end{table} %% GUSTAV'S GUIDE: look at this for how to insert figures, quotes, etc.

%% BONUS STUFF WE MIGHT NEED TO LOOK AT LATER ---------------- VVVVV

%ACKNOWLEDGMENTS are optional
%\section{Acknowledgments}
%This section is optional; it is a location for you
%to acknowledge grants, funding, editing assistance and
%what have you.  In the present case, for example, the
%authors would like to thank Gerald Murray of ACM for
%his help in codifying this \textit{Author's Guide}
%and the \textbf{.cls} and \textbf{.tex} files that it describes.

%
% The following two commands are all you need in the
% initial runs of your .tex file to
% produce the bibliography for the citations in your paper.
\bibliographystyle{abbrv}
\bibliography{references}  % sigproc.bib is the name of the Bibliography in this case
% You must have a proper ".bib" file
%  and remember to run:
% latex bibtex latex latex
% to resolve all references
%
% ACM needs 'a single self-contained file'!
%
%APPENDICES are optional
%\balancecolumns
%\appendix
%Appendix A
%\section{Headings in Appendices}
%The rules about hierarchical headings discussed above for
%the body of the article are different in the appendices.
%In the \textbf{appendix} environment, the command
%\textbf{section} is used to
%indicate the start of each Appendix, with alphabetic order
%designation (i.e. the first is A, the second B, etc.) and
%a title (if you include one).  So, if you need
%hierarchical structure
%\textit{within} an Appendix, start with \textbf{subsection} as the
%highest level. Here is an outline of the body of this
%document in Appendix-appropriate form:
%\subsection{Introduction}
%\subsection{The Body of the Paper}
%\subsubsection{Type Changes and  Special Characters}
%\subsubsection{Math Equations}
%\paragraph{Inline (In-text) Equations}
%\paragraph{Display Equations}
%\subsubsection{Citations}
%\subsubsection{Tables}
%\subsubsection{Figures}
%\subsubsection{Theorem-like Constructs}
%\subsubsection*{A Caveat for the \TeX\ Expert}
%\subsection{Conclusions}
%\subsection{Acknowledgments}
%\subsection{Additional Authors}
%This section is inserted by \LaTeX; you do not insert it.
%You just add the names and information in the
%\texttt{{\char'134}additionalauthors} command at the start
%of the document.
%\subsection{References}
%Generated by bibtex from your ~.bib file.  Run latex,
%then bibtex, then latex twice (to resolve references)
%to create the ~.bbl file.  Insert that ~.bbl file into
%the .tex source file and comment out
%the command \texttt{{\char'134}thebibliography}.
% This next section command marks the start of
% Appendix B, and does not continue the present hierarchy
%\section{More Help for the Hardy}
%The acm\_proc\_article-sp document class file itself is chock-full of succinct
%and helpful comments.  If you consider yourself a moderately
%experienced to expert user of \LaTeX, you may find reading
%it useful but please remember not to change it.
\balancecolumns
% That's all folks!
\end{document}
