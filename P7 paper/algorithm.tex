\subsection{Important Algorithms}
\subsubsection{Interpolation}

Each influence point is placed on the player path. The camera interpolation ($f_{i}$) is a function of the player's position ($P_{pos}$) used to calculate an interpolation between the camera settings ($i1_{cam}$, $i2_{cam}$) of the two influence points closest to the player's position following the paths, with a weight ($w$) based on the positions of the influence points.
\begin{equation}
f_{i} = f(P_{pos})
\end{equation}
The two closest influence points are found by doing a linear search from the player's position along the paths in both directions. The distance travelled in both directions until hitting an influence point is logged ($i1_{dist}, i2_{dist}$).
\begin{equation}
w = i1_{dist}/(i1_{dist} + i2_{dist})
\end{equation}
In the case an influence point was not found in a direction, the total distance travelled in that direction is zero. The artist is further able to manipulate this weight for each camera setting value ($cam_{i}$) through the graph editor. Evaluating the curve at $w$ gives us a new weight $w_{i}$ for each camera setting value. In pseudo code, this gives us the following interpolation for each camera setting:
\begin{equation}
f_{i} = i1_{cs_{i}} * w_{i} + i2_{cs_{i}} * (1-w_{i})
\end{equation}
\subsubsection{Path-based Slider-preview}
To make the slider preview, our solution was to calculate a possible player position ($P_{pos}$) from the given slider-weight ($sw$) and the influence points of the start and end framing ($i_{start}, i_{end}$).
\begin{equation}
P_{pos} = f(sw, i_{start}, i_{end})
\end{equation}
Drawing from the same logic as in the interpolation, a linear search is done from the influence point of the start framing ($i_{start}$) following the paths in both directions, logging the total distance travelled ($TD$) until it hits the influence point of the end framing ($i_{end}$). Each time the search passes another influence point, it logs it ($i_{n}$) together with the distance travelled ($i_{n_{dist}}$). All distances ($i_{n_{dist}}$) are normalized ($i_{n_{distNorm}}$), with the total distance travelled ($TD$) set to 1. The slider-weight ($sw$) now fits in an interval between two influence point distances ($i_{before_{distNorm}}, i_{after_{distNorm}}$). The possible player character position ($P_{pos}$) is then calculated and used as the argument in $f_{i} = f(P_{pos})$.
\begin{eqnarray}
P_{pos} &=& i_{after_{pos}}\left(1 - \left(sw - i_{before_{distNorm}}\right)\right)\nonumber \\
&&\mbox{}+ i_{before_{pos}}\left(sw - i_{before_{distNorm}}\right)
\end{eqnarray}
where $i_{n_{pos}}$ is the world position of $n$ influence point.