\subsection{Discussion}

%Shortcomings and limitations of the test setup
%- Small sample size
%- Artificial setting (unrealistic to learn a new tool in 1 hour)
%- Had to explain the tool before-hand
%- Hard to compare the phases (various lengths, didn't use the same features, different purposes)
%- Hard to compare our tool to other tools ... possible solutions: A) let them re-create camere movements from another game; b) Let them create camera movements in Maya and then do the same in FCT - Compare!
%- People say: we can learn in 2 days!
%- less is more
%- unexpteceded uses
%- they showed interest

Due to several factors, the results gathered from the evaluation are not sufficient to draw statistically significant conclusions. One of these is the low sample size of six participants. Other factors include an artificial setting where the participants had a unrealistic time to both \textit{learn} and be able to \textit{use} FCT. Besides this, they also had to learn the fundamentals of Unity, all within a time period of one hour. For instance, some of the participants believed they would be able to learn the tool properly if they had a little more time available. Ideally, the artists should have enough time to become ``advanced users" before the evaluation even began. This would ensure that the evaluation would look into the FCT's functionality instead of its usability.

One interesting note is that the participants used some of the features in unexpected ways. For instance, one participant used the \textit{aim point} as a pivot point to orbit the camera around. The feature was not designed with this in mind, but it turned out to work rather well. This, together with other instances, indicated that the participants had an interest in learning and using the FCT.

Even though the participants appeared to be able to create the framings that they sketched in the creative phase, it should be noted that they at this point already knew about FCT's strengths and weaknesses. This potentially influenced what they sketched.
When comparing participants' sketches to their implementation (see Figure \ref{fig:Sketching_Framings}), they appear very similar. 
Furthermore, whether or not the participants belong to the participatory design group seems to have little influence on their framings. For instance, Figure \ref{fig:Sketching_Framings} shows sketches and framings created by one participant from the participatory design group and one participant who was not from the participatory design group.
Additionally, the amount of experience in Unity seems to have little impact on the participants' framings, since one of the participants in Figure \ref{fig:Sketching_Framings} had 4-6 years of experience in Unity, while the other had less than 6 months of experience in Unity.

During the evaluation, participants expressed that they would be satisfied with fewer options, e.g., that it was unnecessary to have both the \textit{be the camera} and the \textit{snapshot} feature. The same goes for the multiple ways of previewing. The participants found that the amount of options were maybe too high and cluttered the interface. However, in total, all features were used by the different participants; hence, it would require further investigation to find out what features were redundant.

%Furthermore, there seems to be no significant difference in the framings created by the participants with and without Unity experience. For instance, Figure \ref{fig:Sketching_Framings} shows sketches and framings created by one participant with 4-6 years of Unity experience and one participant with less than 6 months Unity experience (both had 1-3 years Maya experience).

%It should also be noted that the participant with 4-6 years of Unity experience was from the participatory design group, while the other was not. 