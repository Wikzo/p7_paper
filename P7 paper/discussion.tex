\subsection{Discussion}

%Shortcomings and limitations of the test setup
%- Small sample size
%- Artificial setting (unrealistic to learn a new tool in 1 hour)
%- Had to explain the tool before-hand
%- Hard to compare the phases (various lengths, didn't use the same features, different purposes)
%- Hard to compare our tool to other tools ... possible solutions: A) let them re-create camere movements from another game; b) Let them create camera movements in Maya and then do the same in FCT - Compare!
%- People say: we can learn in 2 days!
%- less is more
%- unexpteceded uses
%- they showed interest

Due to several factors, the results gathered from the evaluation are not sufficient to draw statistically significant conclusions. One of these is the low sample size of six test participants. Other factors include an artificial setting where the test participants had a insufficient/unrealistic time to both \textit{learn} and be able to \textit{use} FCT. Besides this, they also had to learn the fundamentals of Unity, all within a time period of one hour. For instance, some of the test participants noted that they believed that they would be able to learn the tool properly if they just had a little more time available. Ideally, the artists should have enough time to become "advanced users" before the evaluation even began. This would ensure that the evaluation would look into the FCT's functionality instead of its usability.

One interesting note is that the participants used some of the features in unexpected ways. For instance, one participant used the \textit{aim point} as a pivot point to orbit the camera around. The feature was not designed with this in mind, but it turned out to work rather well. This, together with other instances, indicated that the participants had an interest in learning and using the FCT.

It is difficult to compare the gathered data from the different phases in training, tasks and creative work. Since all of the phases had different purposes and time spans, it doesn't make sense to compare them against each other. Two suggestions to improve the evaluation of what is possible to achieve with FCT are: a) Let the participants re-create camera movements from other existing games; b) Let the participants create camera movements inside Maya and then re-create the same movements using FCT.

Even though the participants appeared to be able to create the framings that they sketched in the creative phase, it should be noted that they at this point already knew about FCT's strengths and weaknesses. This potentially influenced what they sketched.

During the evaluation, one test participants expressed that they would be satisfied with fewer options, e.g., that it was not necessary to have both the \textit{be the camera} and the \textit{snapshot} feature. The same goes for the multiple ways of previewing. The participant found that the amount of options were maybe too high and cluttered the interface. However, in total, all features were used by the different participants; hence, it would require further investigation to find what features were redundant.

The participants were asked to draw small sketches for how to frame a scene. Later, these were implemented using FCT. Looking at Figure \ref{fig:Sketching_Framings}, it appears that this was possible. Furthermore, there seems to be no significant difference in the framings created by the participants with and without Unity experience. For instance, Figure \ref{fig:Sketching_Framings} shows sketches and framings created by one participant with 4-6 years Unity experience and one participant with less than 6 months Unity experience (both had 1-3 years Maya experience).