\section{Participatory Design Findings}
The user group of the tool is a team of six third year students from The Animation Workshop in Viborg, Denmark who are developing a 3D point-and-click adventure game \textbf{(INSERT REFERENCE TO GAME GENRE)}.  The team is a mixture of students from two different educations at the school; Computer Animation and Computer Graphics \textbf{(INSERT REFERENCE TO TAW WEBSITE)}. The students will from here on be referred to as artists.  All artists are familiar with making animation films with Autodesk Maya, however it is the first time they are making a video game. Since the start of the project, the students have been educated in working with the Unity game engine.
A requirement for the camera tool as requested by the team, was that it should work with a path system, which is used for player navigation. Furthermore, it was requested that the camera should give the artists the freedom to frame the scene as they want, in order to focus on different elements in the scene.
To create a tool that takes the artists' current skills and experience into consideration, thereby easing their learning curve, techniques from participatory design were used.

\subsection{Participatory Design Activities}
Four different activities using a combination of participatory design techniques were conducted with four artists from the team who had an interest in working with the camera tool. All activities were combined with open interviews as well as video recorded for further analysis.

The first activity, was an observation (INSERT REFERENCE) of the artists working in Maya and Unity. The artists were asked to think-aloud (INSERT REFERENCE) while performing simple tasks relevant to camera work in Maya and Unity. The purpose of this activity was to gain firsthand experience with the artists' current work practices in Maya as well an understanding of how familiar they were with Unity. The activity took 10-15 minutes per person.

Second, the artists were asked to list and sketch the features that they would like to see the most in the camera tool. This activity took inspiration from the 'Freehand Drawing' technique (INSERT REFERENCE). The purpose of the activity was to get a foundation for which tasks the artists wanted to be able to do. The activity took 10-15 minutes per person.   

After sketching, the artists were asked to visualize the interface of the camera tool by arranging a set of pre-made buttons and windows out of paper based on Unity's interface. The activity took inspiration from the 'Collage' technique (INSERT REFERENCE) and its purpose was to get a foundation for the interface design. This activity took 10-15 minutes per person.

The final activity was made after the first design iteration...

\subsection{Findings}
Through the participatory design exercises, we found that the artists were very familiar with Autodesk Maya \cite{MayaSource}. To ease the learning curve, we decided to design the tool with the overall structure of Maya in mind. This doesn't mean that the tool should directly copy how Maya looks and works, but that, when designing the camera tool, we should keep their established mental models \cite{mentalModels} in mind. Since the artists have been working with Maya and other 3D modelling software, they have a certain understanding and expectation of the animation process. As a designer, you create conceptual models for the system. Therefore, it's important to be aware of the users' pre-existing mental models. Another key finding was the fact that most of the artists were working with two computer monitors. This enables them to have the main interface on one monitor, while the other can show secondary windows such as graph editors or an additional preview window showing the camera's point of view.