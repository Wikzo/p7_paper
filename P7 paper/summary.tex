\section{Summary and Future Work}
We have defined the basic requirements needed for a framing-based camera tool built specifically for a game where the player character's movement is restricted to a linear path. From our evaluation, our tool successfully let the user design and create a camera set-up and proved versatile enough to accommodate creative freedom and even usages not originally envisioned. 

It is difficult to compare the gathered data from the different phases in training, tasks and creative work. Since all of the phases had different purposes and time spans, it doesn't make sense to compare them against each other. Two suggestions to improve the evaluation of what is possible to achieve with FCT are: a) Let the participants re-create camera movements from other existing games; b) Let the participants create camera movements inside Maya and then re-create the same movements using FCT.

A limitation of this project was the study of creativity, i.e. how the artists were able to get an idea, sketch it out on paper and then implement it with FCT. It raises the question how to judge creativity and quality. One approach is to let professionals in the field of camera animation judge the end results of the framings anonymously \cite{sadeghi_artist_2010}. We encourage more research into this field.