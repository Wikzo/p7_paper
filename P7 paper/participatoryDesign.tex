\section{Participatory Design}
Participatory design (PD) has been defined as the participation of users in the design process of a system that is to be implemented in an organization (INSERT REF). Here, users correspond to workers in the organization. By involving users in the design, their skills, experiences, and interests are taken into consideration, thereby increasing the likelihood that the system will be useful to them.

According to PD researchers, it is important there is an active cooperation between the designer and the user(INSERT REF). This results in the designer gaining knowledge of the user's current work practices, and the user gaining knowledge of the technology to be used. It is also important that users take an active part in the analysis of needs, selection of technology, design and prototyping, as well as organizational implementation.

REMEMBER TO EXPLAIN WHY WE USE 4 OUT OF 6 ARTISTS IN PD FINDINGS

\subsection{Methods and techniques}
General techniques
Method-specific techniques

\subsubsection{Sketching}
BENJAMIN.Sketching was a part of

\subsubsection{Collage}
BENJAMIN.Collage is more restrictive ...