\subsubsection{Procedure}
Each participant was tested individually in a meeting room (see Figure \ref{fig:tt}). Besides the participant, a facilitator and an observer was also present in the room. The test was split into seven parts. The procedure was the same for all participants. The evaluation sessions lasted for approximately 50-60 minutes each.

\begin{enumerate}[noitemsep,nolistsep]
\item Consent, Demographic and Introduction
\item Hands-on with Camera Path Animator
\item Basic Navigation in Unity
\item Training with FCT
\item Tasks with FCT
\item Creative use of FCT
\item Post-test Questionnaire
\end{enumerate}

\textbf{Consent, Demographic and Introduction}\\
The participants were given a brief introduction and presented with a consent form to allow video recording. The participants started by answering a demographic questionnaire. 

\begin{figure}[htbp]
\centering
\includegraphics[width=0.3\textwidth]{Pics/test_setup}
\caption{The test participant used two monitors while working with FCT.}
\label{fig:tt}
\end{figure}

\textbf{Hands-on with Camera Path Animator}\\
The test participants were introduced to a small demo of the Camera Path Animator \cite{unity_camTool} (see Section \ref{relatedWork}). They were told to move the player character around and notice how the camera behaved accordingly. This was to give the participants a context for following test of FCT, as well as introduce them to the concept of camera movement in an interactive environment.

\textbf{Basic Navigation in Unity}\\
To ensure that all participants had a basic understanding of how to navigate in Unity, the facilitator gave a brief introduction to the basic functionality of Unity. This included a short description of the essential windows in Unity (scene view, game view, hierarchy and the inspector), as well as how to move and rotate objects. The participants were also instructred how to move the scene view camera around in Unity. They were also told about Unity's \textit{Flythrough Mode} \cite{unity_flyMode}. To ensure that they actually understood how to navigate around, they were asked to move the camera to three specific locations in the scene. For instance, they were told to place the camera on top of the green cube seen in Figure \ref{fig:sceneAll}.

This phase lasted around 2-5 minutes.

\textbf{Training with FCT}\\
Before trying FCT, the facilitator gave a short oral introduction of the framing concept by showing a printout of Figure \ref{fig:framingConceptNew}. Afterwards, the participants were asked to open FCT in Unity. Here, the facilitator went through all of the major features in a semi-structured way. Each feature was explained one a a time, and the test participant tried them one at a time before moving to the next.

A path was already defined in the scene. The test participants learned about how to place and adjust a framing. They learned how to create and move influence points. They then went through features such as changing the position and rotation of a camera using the different methods, i.e. changing the values by hand; using the \textit{be the camera} feature; using the \textit{snapshot} feature; and using the \textit{aim point}. In order to preview their changes, the participants were shown the \textit{interactive preview} and \textit{slider preview}. Afterwards, they were told about the graph editor, as well as miscellaneous smaller features.

This phase lasted around 15-20 minutes.

\textbf{Tasks with FCT}\\
After the participants had tried out all of the major features in FCT, they were handed handed a piece of paper with five tasks. These tasks required that the participants understood the concept about framings and knew how to create them using the tool. For each task, they were asked to create a new framing.

\begin{enumerate}[noitemsep,nolistsep]
\item Make the camera's field of view change.
\item Make the camera tilt upwards.
\item Make the camera look at the tall pink cylinder.
\item Make the camera go from a low perspective to a bird's-eye view.
\item Change the interpolation of one of the previous assignments by changing the animation curve.
\end{enumerate} 

The chosen tasks reflects five common features and tasks that's often used when setting up a framing and a camera interpolation. The participants had to solve the tasks themselves; the facilitator only intervened when the participant was struggling with something, asked a question, or unforeseen errors occurred with the tool.

This phase lasted around 15-20 minutes.

\textbf{Creative use of FCT}\\
After this, the participant was introduced to a level with a modelled environment. They were then tasked to envision and sketch two ways for the camera to move as the play character moved through this environment, when they had two ideas, they were tasked to implement both of these using our camera tool. The facilitator remained as neutral as possible for this part of the test, but still intervened if they participant was struggling or encountered things like bugs.



\begin{figure*}[htbp]
\centering
\includegraphics[width=0.8\textwidth]{Pics/Sketching_Framings}
\caption{Test participants first sketched their ideas on paper; then they implemented it using FCT.}
\label{fig:Sketching_Framings}
\end{figure*}

This phase lasted around 5-10 minutes.

\textbf{Post-test Questionnaire}\\
Finally, the test participant answered a short questionnaire about their experience with FCT.