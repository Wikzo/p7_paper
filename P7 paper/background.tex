\section{Background Knowledge}
%Theory behind virtual cameras (will mostly be based on \cite{haigh-hutchinson_real-time_2009}.
%Describe different camera types in games (path-based; first-person vs third-person; on-rails vs. manual; 2D vs. 3D, etc.).
%Interpolation
%Easing (filtering)
%Etc.

There are many ways to design camera motions in games. Fundamentally, one can distinguish between \textit{cinematic sequences} and \textit{interactive gameplay}. Normally, these two are considered mutually exclusive, because cinematic sequences per definition is non-interactive \cite{haigh-hutchinson_real-time_2009}. However, it is also possible to mix those two, so the camera can dynamically adapt to certain things happening in the game, e.g., game events and player input. This means that a sequence does not have to be viewed in exactly the same way every time. For example, depending on the player's movement, the camera can change accordingly.

%Sometimes it might be necessary to put certain restrictions on where the player can move. An example of this could be a special "boss battle" where the player is confined in a restricted area. Typically, the camera would zoom out and focus on specific parts of this boss enemy (e.g. a weak point). The camera dynamically frames the scene in such a way so that the player and the enemy are visible at all times \cite{haigh-hutchinson_real-time_2009}.

Games often require the ability to replay previous sequences of the gameplay. This can be used to replicate certain events to re-create the motion and visual state of objects in a scene \cite{haigh-hutchinson_real-time_2009}. It can be achieved by recording the rendering state of objects on a set amount of frames, and then use \textit{interpolation} to calculate the state of said objects. Interpolation is a method of inserting intermediate values into a set of data and makes it possible to take sampled data and generate new points in between \cite{haigh-hutchinson_real-time_2009}. Replaying of this data can be referred to as \textit{keyframing}. Keyframing of camera data requires position and orientation of the camera, together with a time interval between the samples \cite{haigh-hutchinson_real-time_2009}.