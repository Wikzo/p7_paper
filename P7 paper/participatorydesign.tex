\subsection{Participatory Design}
An initial mockup of the tool's interface and functionality was created internally. The mockup had the interface all gathered in one window to keep the design minimalist. However, in search of a better design, we included the end-users in the process going forward. Beskriv mockup mere i dybde?

To ground our design work, we were interested in learning more about our users and how they envisioned the finished product. Therefore, three participatory exercises were conducted. The first of which was done to know their general workflow when working with cameras in the tools they use (Maya and Unity). Secondly, we wanted to know which features they requested and how they envisioned the finished tool. Finally, they tried using a prototype of the camera tool based on their initial design and feature requests.

Note that the following three exercises has been video documented, except for one artist and half of the footage from another artist in the first exercise.

Exercise 1: Knowledge of tools. 
An artist and a facilitator sat down in front of the artist's work area. The facilitator then asked the artist to show him how he would animate a camera around an object in Maya, this included keyframing and how to change the camera settings for each keyframe. This was done to get an idea of how they usually work. If our camera tool should be successful at helping these artists, it should mimic some of the same behavior that Maya has. After this the facilitator and artist switched to Unity where he was tasked with moving around in the scene and then to create an object (cube, sphere) and position the camera. The point of this was to get an understanding of how skilled they are at using unity - moving around the scene, rotating and translating objects etc. This was done with another two artists as well. The first section of the exercise, the Maya part, was the same for all three artists, but the questions and tasks changed in the last section as the facilitator gained new knowledge about their skill level.

It was discovered that 
It was discovered that none of the artists knew how to use a standard Unity feature that lets the player fly around with the scene camera as if they were playing a first-person game. The artists was very excited at the discovery of this feature. One artist perceived the standard way of moving around in Unity as confusing, while another stated that the way of moving the camera is exactly like in Maya. After testing this ourselves, we concluded that the movement controls in Maya and Unity are exactly the same (except for the "fly around"-mode in Unity) and can therefore say it should be no problem for the artists to move around in Unity. 

Exercise 2: Feature list and paper prototyping.
The artists was given a blank sheet of paper and a pen and was tasked to draw or write about the camera tool as they envisioned it in a free-form approach. They then explained what they did to a facilitator and they had a discussion about it. The amount of EDIT THIS TO SOMETHING varied from artist to artist, from one who wrote different settings for the camera to others who drew an entire interface. This was done with four artists in total.

After this, the same four artists were placed at a desk, one by one, where a variety of labels was laid out in front of them. These labels were marked as different sliders, buttons, windows, field parameters, camera settings, etc. Some of  labels was based on the feature lists they created in the first part of this exercise. See figure 1 for the piles of labels at their disposal. The artists was tasked with building an interface for the camera tool using these labels as building blocks.

All artists wanted basic features like translation, rotation, field of view of the camera (in mm) as well as a curve editor to change the interpolation between two cameras. After these features all the artists diverted in their designs. All the designs were discussed internally and main points and ideas was extracted from them. These findings was used as the foundation of the tool's design, but features not mentioned by the artists was implemented if we found them meaningful, since we are the designers.

A slider that could be used to interpolate between two camera framings in a preview window is an example of an idea that was discarded by us. The artist wanted two preview windows of the camera framing as well as a preview window of the interpolation which the slider would manipulate. After discussing it internally, we concluded that this idea would make the interface too elaborate, but the idea of being able to preview your changes quickly was kept.

FIGURETEXT:
Figure 1: The labels that the artists used to create an interface. Blank sheets of papers were also at their disposal if they did not find what they wanted.