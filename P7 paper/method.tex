\subsection{Method} \label{method}

\subsubsection{Participants}
Four male and two female third-year students at The Animation Workshop participated in the test of our camera tool. Three males was from the \textit{Feels} production and one male and the two females was other third-year students no associated with the production. Four participants reported to have between one and three years of experience with Autodesk Maya, another had between four and six years of experience and the last had seven or more years of experience. Only one participant had more than one year of experience with Unity, another had between six and twelve months of experience with it, the rest reported to have less than six months of experience.

\subsubsection{Procedure}
Each participant was tested individually in a small meeting room. Besides the participant, a facilitator and an observer was also present in the room. The participant were given a brief introduction and was subsequently presented with a consent form. With their consent, the test began including video recording. The participants started by answering a demographic questionnaire on a laptop. After this they were introduced to a demo of the Camera Path Animator (introduced in Section \ref{relatedWork}) \cite{unity_camTool}. This was to give the participant context for the test and to introduce them to the concept of camera behaviour in an interactive environment. 

The participants and facilitator sat down at another laptop with a second monitor connected. Here the facilitator gave the participant a basic introduction to Unity to ensure that all participants had no problem of navigating and manipulating the workspace. The participant was tasked to move the camera to three specific locations in the environment to ensure they felt comfortable in the workspace. After this the facilitator introduced the camera tool to the participant, first he explained the concept with pen and paper, then he opened the tool in Unity and explained its features and functionalities. The participant was asked to try each feature as they were introduced, e.g. when the facilitator explained how to place influence points and how to connect them, the participant was asked to try it for themselves. When all features were explained, and the facilitator felt that the participant had a good grasp on the tool, the participant was handed a piece of paper with 5 tasks such as \textit{"Make the camera's field of view change"} and \textit{"Make the camera go from a low perspective to bird's eye view."} listed on it. The participant had to solve the tasks themselves, the facilitator only intervened when the participant was struggling with something or at unforeseen occurrences (e.g. bugs). After this, the participant was introduced to a level with a modeled environment. They were then tasked to envision and sketch two ways for the camera to move as the play character moved through this environment, when they had two ideas, they were tasked to implement both of these using the our camera tool. The facilitator remained as neutral as possible for this part of the test, but still intervened if they participant was struggling or encountered things like bugs.

Finally, the participant went back to the first laptop and answered a post-test questionnaire. The participant was thanked and the test ended.

\subsubsection{Materials}
Three scenes was constructed for the test. The first was a environment with simple geometry scene. The second was used in the first part of the test when participants were introduced to the camera tool and had to complete 5 tasks. The third level was for the creative part of the test when participants had to envision the camera movement in an environment and then implement it. 

\begin{figure}[htbp]
\centering
\includegraphics[width=0.45\textwidth]{Pics/sceneAll}
\caption{1) The participant was instructed to move the camera to the top of the green cube, between the purple pillars and to the bottom of the red cube and then rotate the camera to look at the purple pillars. 2) A flat plane with simple geometry to make it easier to see differences when changing camera settings. 3) A mountainous environment with a staircase attacged to the mountainside.}
\label{fig:sceneAll}
\end{figure}
