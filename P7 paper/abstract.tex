\begin{abstract}
Introduction
The goal of this project was to create a camera tool in Unity, which can be used by non-programmers. The system was be designed specifically for a team of six third year students of The Animation Workshop (TAW) in Viborg, who are developing a 3D point-and-click game in Unity for the iPad. The camera tool uses interpolation based on player position [1].

Participatory Design and Experiment
The students of TAW were included in the design process. In order to get a better understanding of what the tool should contain, three participatory exercises were conducted with the students of TAW [2]. Using this as the foundation, the camera tool was designed and implemented to accommodate the most essential requests. Later, an experiment was conducted with six test participants. The goal was to investigate to what extent the students were able to use the tool to create camera framings inside the game. Since all of the students regularly use Maya, it was important for them to be able to transfer their skills into the camera tool. During the experiment, a facilitator described the key features of the tool. Afterwards, the participants had to solve tasks that showed that they understood the tool. Finally, they were asked to use the tool in creative ways to create two artistic framings in a simple scene.

Findings 
All of the test participants managed to complete the given tasks without any major problems. Generally, they felt that they had freedom to use the tool in ways that made sense to them. The tool provided multiple possibilities, such as changing position and rotation, field of view, and camera easing. However, some participants expressed concerns that the tool might provide an overwhelming amount of methods to complete the same tasks differently.

Conclusion
The tool successfully gave the participants the ability to easily set up cameras both quickly and nonrestrictively. The tool proved to be easy enough to be used for quick sketching and broad enough for final setup. Further user experiments are needed to make the interface more streamlined and intuitive to ease workflow. For a fully featured tool, it would further need to be able to handle forked paths, cuts, cutscenes and camera effects.

\end{abstract}