\begin{abstract}
Camera control in games is important, for example to create cinematic graphic effects or to guide the attention of the player. Unlike the camera in pre-rendered movies, a game camera should adapt to the player's movements. This paper describes the design of a tool that allows artists to frame the camera in games where the player moves along a pre-defined path. To this end, we used participatory design methods to understand how artists typically work with computer animations and found that they prefer to work with keyframing animation. This concept has been incorporated into the proposed Framing-based Camera Tool (FCT) in the form of framings. A framing consists of an influence point and a group of camera settings. Artists are able to define a group of framings by adjusting the position, orientation and field of view of the camera. Then, the game's camera interpolates between the framings automatically based on the player's position.

\textbf{(Something about the results from the evaluation)}
\end{abstract}