\begin{abstract}
It is important to be able to frame a virtual camera in a game, e.g., to create a certain mood or to tell the player where he should be heading next. Unlike movies, where the camera moves linearly, a game camera should adapt to the player's movements. This paper sets out to investigate how to create a tool that allows artists to frame the camera in games where the player moves along a pre-defined path. To approach this, we used participatory design methods to understand how artists typically work with animations. We found that they like to work with keyframing animation. This concept has been incorporated into the Framing-based Camera Tool (FCT) in the form of framings. A framing consists of an influence point and a group of camera settings. Artist are able to define a group of framings by adjusting the position, orientation and field of view of the camera. Then, depending on where the player is located, the game's camera will interpolate between the framings automatically.
\end{abstract}