\subsubsection{Data Analysis}
The video recordings from the experiment was coded internally at a later date in pairs of two. The two pairs coded one participant in unison and then split up and worked in parallel. Repeated ideas, events and observations was tagged with certain tags. The tags used are \textit{confusion}, \textit{exploration}, and \textit{problem}.

An event was tagged with \textit{confusion} if a participant was confused that a feature didn't work as they expected or struggled doing what they wanted/envisioned (e.g. did not connect influence points; unsure why interpolation didn't not function properly), 
%\textit{exploration} means that a participant explored the interface or features of the camera tool by themselves without prompting them or if they asked questions about the features (e.g. \textit{"What consequence does it have if the points are not connected? Would that be a cut?"}), 
%\textit{observation} is an (subjective) observation made internally about the participant and their behavior or approach to the tasks (e.g. no problem navigating in Unity), 
\textit{problem} means a participant got stuck or encountered a problem so they could not progress (e.g. tried changing position of camera while Aim Point was active. The camera jumped around seemingly random in the scene and moved to an unwanted position because of it.)