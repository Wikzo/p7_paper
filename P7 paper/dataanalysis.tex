\subsubsection{Data Analysis}
The video recordings from the evaluation were coded internally in pairs of two. Repeated ideas, events and observations were tagged \textit{confusion} or \textit{problem}.

If a feature didn't work as the participant expected, the event was tagged as \textit{confusion}. If a participant struggled with what they wanted/envisioned to achieve, e.g. not being able to connect influence points, they were also tagged as \textit{confusion}.

The \textit{problem} tag was given if a participant got stuck or encountered a problem so severe they could not progress. For instance, if they tried changing the position of a camera with the \textit{be the camera} feature while the \textit{aim point} was active. This made the camera jump around seemingly random in the scene, sometimes moving to unwanted positions.

%An event was tagged \textit{confusion} if a participant was confused, e.g., a feature didn't work as expected. The \textit{confusion} keyword also include whenever somebody struggled with what they wanted or envisioned to do, e.g., not being able to connect influence points or unsure why the interpolation didn't function properly. 
%\textit{exploration} means that a participant explored the interface or features of the camera tool by themselves without prompting them or if they asked questions about the features (e.g. \textit{"What consequence does it have if the points are not connected? Would that be a cut?"}), 
%\textit{observation} is an (subjective) observation made internally about the participant and their behavior or approach to the tasks (e.g. no problem navigating in Unity), 
