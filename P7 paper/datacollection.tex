\subsection{Questionnaire}
The participants had varied experience with Autodesk Maya, ranging from six months to six years. In Unity, their experience ranged from none at all to more than a year.

In the questionnaire, the participants rated their agreement with a list of statements using a 5-point Likert scale ranging from "strongly disagree" to "strongly agree". 

%While the test participants tried the FCT, they were recorded  their two monitors were. Additionally, a webcam recorded the face of the participants, together with the audio. An observer wrote notes during the evaluation. For the questionnaire, the participants rated themselves on the following statements using a 5-point Likert scale ranging from "strongly disagree" to "strongly agree". 
%\begin{itemize}[noitemsep,nolistsep]
%\item \textit{I felt empowered using this tool.}
%\item \textit{I felt restricted using this tool.}
%\item \textit{I felt I got the tasks done quickly.}
%\item \textit{I felt the tool allowed me to complete the tasks well.}
%\end{itemize}

When asked if the participants \textit{"felt empowered using the tool"}, four participants \textit{agreed}, while the remaining two \textit{strongly agreed}. When asked if they \textit{"felt restricted using the tool"}, five participants \textit{disagreed} while the remaining participant \textit{strongly disagreed}.

When asked they if they felt that they \textit{"got the tasks done quickly"}, four participants \textit{agreed} and two \textit{strongly agreed}. To find out how effective the participants felt using our tool, we asked them if they \textit{"felt the tool allowed them to complete the tasks well"}. Three participants \textit{agreed} and three \textit{strongly agreed}.

%To find out how efficient the participants felt using the FCT, we asked them if they felt they \textit{"got the tasks done quickly"}. Four participants \textit{agreed} and two \textit{strongly agreed}. To find out how effective the participants felt using our tool, we asked them if they \textit{"felt the tool allowed them to complete the tasks well"}. Three participants \textit{agreed} and three \textit{strongly agreed}.

Furthermore, they were asked whether they preferred the \textit{be the camera} or the \textit{snapshot} feature for adjusting cameras. Four named\textit{ be the camera} and two named \textit{snapshot}. 
They were also asked how useful they thought the preview features were. The response choices were "not useful", "somewhat useful", "very useful", "didn't use any of them", and "don't know". Five participants found them \textit{very useful}, and one found it \textit{somewhat useful}.
Finally, the participants were asked what their overall favorite feature was. Three participants noted the \textit{be the camera} feature as their overall favourite feature, while the \textit{slider preview} was the favourite for the other three participants. 

%When asked about the preview features, five participants found them \textit{very useful}, and one found it \textit{somewhat useful}. The participants were also asked whether they preferred the \textit{be the camera} or the \textit{snapshot} feature. Four named\textit{ be the camera} and two named \textit{snapshot}. Furthermore, three participants noted the \textit{be the camera} feature as their overall favourite feature, while the \textit{slider preview} was the favourite for the other three participants. 

%Furthermore, they were asked whether they preferred the \textit{be the camera} or the \textit{snapshot} feature for adjusting cameras. They were also asked how useful they thought the preview features were. The response choices were "not useful", "somewhat useful", "very useful", "didn't use any of them", and "don't know". To highlight the best features in the tool, the participants were asked to name their favourite feature from a list of 7.

%When asked about the preview features, five participants found them \textit{very useful}, and one found it \textit{somewhat useful}. The participants were also asked whether they preferred the \textit{be the camera} or the \textit{snapshot} feature. Four named\textit{ be the camera} and two named \textit{snapshot}. Furthermore, three participants noted the \textit{be the camera} feature as their overall favourite feature, while the \textit{slider preview} was the favourite for the other three participants. 