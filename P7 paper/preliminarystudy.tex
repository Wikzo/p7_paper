\section{Preliminary Study}
We did bla bla bla ...

\subsection{Participants}
	Interviews (Knapnok Games, Unity Studios)
	Online questionnaires (27 game developers)
	Observations (4-5 students at TAW)
	
	The interview sessions involved a total of three people from two game studios in Denmark; Project Lead at Knapnok Games in Copenhagen and Lead Designer and Lead Programmer at Unity Studios in Aarhus.
The online questionnaire received 27 submissions from people with a median professional experience of 3 years (SD 2.67).
Lastly, the participatory paper prototyping was done on 4 students at TAW. (from the team?!)

	
\subsection{Methods}
		Interview design (semi-structured)
		Questionnaire design
		Observations (from meetings, discussions)
		Paper prototyping
		Participatory design
		
		The interviews were conducted at the respective companie' offices using a semi-structured approach. The interviews was audio recorded and notes were taken during the interview. We were offered to see the tools they use in production in practice. The session at KnapNok Games took one hour and two hours at Unity Studios.
The questionnaires was published on relevant game development communities. (unity forum, tigsource, 3Dboss.com, game dev facebook groups and twitter). The questions in the questionnaire and the semi-structured interview was designed to be case-specific (hendrik bog) in hope of gathering more detailed information. Both the interviews and questionnaire was analysed with using grounded theory
The paper prototyping was a participatory design - dialogue-based prototyping session. The participants wrote down their requirements while thinking aloud. They then designed their "dream interface" using paper blocks. They expressed their decisions and ideas while doing so. A facilitator would ask why they needed certain elements and why it was chosen to structure it as they did.


\subsection{Findings}
	Frequently mentioned points
	
	The two game studios that were interviewed both use Unity in their development and mentioned that it is most efficient if everyone on the development team is able to use Unity. This allows team members non-programmers (e.g. artists and designers) to implement assets themselves as well as tweak parameters and build functionality using tools made by the programmers. This way, the amount of interactions between programmers and non-programmers are minimized, and non-programmers can take on more tasks.
Among the tools made for non-programmers was 
Points derived from the questionnaire mentions the more tweakable  your design is, the better, i.e. changing variables without going into the code in a text editor. However, this can also be of nuisance as the number of accessible variables can confuse them. A tool also have to be presented clearly. Finally it is important that the designers know the tools they are working with.
During the paper prototyping, the participants continuously mentioned Maya (the software package they're most accustomed to) when trying to explain how they had envisioned the camera tool working; they wanted something close to Maya.
